% Created 2017-03-05 Sun 22:51
% Intended LaTeX compiler: pdflatex
\documentclass[a4paper,11pt]{article}
\usepackage[utf8]{inputenc}
\usepackage[T1]{fontenc}
\usepackage{graphicx}
\usepackage{grffile}
\usepackage{longtable}
\usepackage{wrapfig}
\usepackage{rotating}
\usepackage[normalem]{ulem}
\usepackage{amsmath}
\usepackage{textcomp}
\usepackage{amssymb}
\usepackage{capt-of}
\usepackage{hyperref}
\usepackage[margin=1.2in]{geometry}
\usepackage{setspace}
\singlespacing
\usepackage{parskip}
\usepackage{amsthm}
\usepackage{amsmath}
\newcommand{\dx}{\mathrm{d}}
\newcommand{\var}{\mathrm{var}}
\newcommand{\cov}{\mathrm{cov}}
\newcommand{\corr}{\mathrm{corr}}
\newcommand{\pr}{\mathrm{Pr}}
\date{}
\title{Homework Set 1}
\hypersetup{
 pdfauthor={},
 pdftitle={Homework Set 1},
 pdfkeywords={},
 pdfsubject={},
 pdfcreator={Emacs 25.1.1 (Org mode 9.0.3)},
 pdflang={English}}
\begin{document}

\maketitle
All questions are from the end-of-chapter exercises. The question
numbers refer to those in the book. The due time is \textbf{March 13\(^{\text{th}}\)}.

\begin{description}
\item[{2.6}] The table below gives the joint probability distribution
between employment status and college graduation among those
either employed or looking for work (unemployed) in the
working age U.S. population for 2008.

\begin{center}
\begin{tabular}{lrrr}
 & Unemployed (Y=0) & Employed (Y=1) & Total\\
\hline
Non-college grads (X=0) & 0.037 & 0.622 & 0.659\\
College grads (X=1) & 0.009 & 0.332 & 0.341\\
\hline
Total & 0.046 & 0.954 & 1.000\\
\end{tabular}
\end{center}

\begin{description}
\item[{a.}] Compute \(E(Y)\).
\item[{b.}] The unemployment rate is the fraction of the labor force that is
unemployed. Show that the unemployment rate is given by \(1-E(Y)\).
\item[{c.}] Calculate the unemployment rate for (i) college graduates and (ii)
non-college graduates
\item[{d.}] A randomly selected member of this population reports being
unemployed. What is the probability that this worker is a college
graduate? A non-college graduates?
\item[{e.}] Are educational achievement and employment status independent?
Explain.
\end{description}
\end{description}


\begin{description}
\item[{2.10}] Compute the following probabilities:
\begin{description}
\item[{a.}] If Y is distributed \(N(1, 4)\), find \(\pr(Y \leq 3)\).
\item[{b.}] If Y is distributed \(N(3, 9)\), find \(\pr(Y > 0)\).
\item[{c.}] If Y is distributed \(N(50, 25)\), find \(\pr(40 \leq Y \leq 52)\).
\item[{d.}] If Y is distributed \(N(5, 2)\), find \(\pr(6 \leq Y \leq 8)\).
\end{description}
\end{description}


\begin{description}
\item[{2.13}] \(X\) is a Bernoulli random variable with \(\pr(X = 1) = 0.99\),
\(Y\) is distributed \(N(0, 1)\), \(W\) is distributed \(N(0,
          100)\), and \(X\), \(Y\), and \(W\) are independent. Let \(S=XY +
          (1-X)W\). (That is, \(S=Y\) when \(X=1\), and \(S=W\) when \(X=0\))
\begin{description}
\item[{a.}] Show that \(E(Y^2) = 1\) and \(E(W^2) = 100\)
\item[{b.}] show that \(E(Y^3) = 0\) and \(E(W^3) = 0\). (Hint: What is the
skewness for a symmetric distribution?)
\item[{c.}] Show that \(E(Y^4) = 3\) and \(E(W^4) = 3 \times 100^2\). (Hint: Use
the fact that the kurtosis is 3 for a normal distribution.)
\item[{d.}] Derive \(E(S), E(S^2), E(S^3), \text{ and } E(S^4)\). (Hint: Use
the law of iterated expectations conditioning on \(X=0 \text{ and
     } X=1\))
\item[{e.}] Derive the skewness and kurtosis for \(S\).
\end{description}
\end{description}


\begin{description}
\item[{2.23}] This exercise provides an example of a pair of random
variables \(X\) and \(Y\) for which the conditional mean of \(Y\)
given \(X\) depends on \(X\) but \(\corr(X, Y)=0\).

Let \(X\) and \(Z\) be two independently distributed standard
normal random variables, and let \(Y = X^2 + Z\).
\begin{description}
\item[{a.}] Show that \(E(Y|X) = X^2\)
\item[{b.}] Show that \(\mu_Y = 1\).
\item[{c.}] Show that \(E(XY) = 0\). (Hint: Use the fact that the odd
moments of a standard normal random variable are all zero.)
\item[{d.}] Show that \(\cov(X, Y) = 0\) and thus \(\corr(X, Y) = 0\)
\end{description}
\end{description}


\begin{description}
\item[{2.26}] Suppose that \(Y_1, Y_2, \ldots, Y_n\) are random variables with a
common mean \(\mu_Y\), a common variance \(\sigma^2_Y\), and the
same correlation \(\rho\) (so that the correlation between
\(Y_i\) and \(Y_j\) is equal to \(\rho\) for all pairs \(i\) and \(j\)
, where \(i \neq j\).)
\begin{description}
\item[{a.}] Show that \(\cov(Y_i, Y_j) = \rho\sigma^2_Y\) for \(i \neq j\).
\item[{b.}] Suppose that \(n=2\). Show that \(E(\overline{Y})=\mu_Y\) and
\(\var(\overline{Y}) = \frac{1}{2}\sigma^2_Y +
          \frac{1}{2}\rho\sigma^2_Y\).
\item[{c.}] For \(n \geq 2\), show that \(E(\overline{Y}) = \mu_Y\) and
\(\var(\overline{Y}) = \sigma^2_Y/n +
          [(n-1)/n]\rho\sigma^2_Y\).
\item[{d.}] When \(n\) is very large, show that \(\var(\overline{Y})
          \approx \rho\sigma^2_Y\).
\end{description}
\end{description}


\begin{description}
\item[{3.3}] In a survey of 400 likely voters,215 responded that they
would vote for the incumbent and 185 responded that they
would vote for the challenger. Let \(p\) denote the fraction of
all likely voters who preferred to incumbent at the time of
the survey, and let \(\hat{p}\) be the fraction of survey
respondents who preferred the incumbent.
\begin{description}
\item[{a.}] Use the survey results to estimate \(p\).
\item[{b.}] Use the estimator of the variance of \(\hat{p}\),
\(\hat{p}(1-\hat{p})/n\), to calculate the standard error of
your estimator.
\item[{c.}] What is the p-value for the test \(H_0: p=0.5\) v.s. \(H_1: p
          \neq 0.5\)?
\item[{d.}] What is the p-value for the test \(H_0: p=0.5\) v.s. \(H_1: p > 0.5\)?
\item[{e.}] Why do the results from (\textbf{c}) and (\textbf{d}) differ?
\item[{f.}] Did the survey contain statistically significant evidence
that the incumbent was ahead of the challenger at the time
of the survey? Explain.
\end{description}
\end{description}


\begin{description}
\item[{3.9}] Suppose that a lightbulb manufacturing plant produces bulbs
with a mean life of 2000 hours and a standard deviation of
200 hours. An inventor claims to have developed an improved
process that produces bulbs with a longer mean life and the
same standard deviation. The plant manager randomly selects
100 bulbs produced by the process. She says that she will
believe the inventor's claim if the sample mean life of the
bulbs is greater than 2100 hours; otherwise, she will
conclude that the new process is no better than the old
process. Let \(\mu\) denote the mean of the new process. Consider
the null and alternative hypothesis \(H_0: \mu = 2000\)
v.s. \(H_1: \mu > 2000\).
\begin{description}
\item[{a.}] What is the size of the plant manager's testing procedure?
\item[{b.}] Suppose the new process is in fact better and has a mean
bulb life of 2150 hours. What is the power of the plant
manager's testing procedure?
\item[{c.}] What testing procedure should the plant manager use if she
wants the size of her test to be 5\%?
\end{description}
\end{description}


\begin{description}
\item[{3.10}] Consider the estimator \(\tilde{Y}\), defined in Equation
(3.1). Show that (a) \(E(\tilde{Y}) = \mu_Y\) and (b)
\(\var(\tilde{Y}) = 1.25\sigma^2_Y/n\).

Equation (3.1) is
\[ \tilde{Y} = \frac{1}{n}(\frac{1}{2}Y_1 + \frac{3}{2}Y_2 +
          \frac{1}{2}Y_3 + \frac{3}{2}Y_4 + \cdots +
          \frac{1}{2}Y_{n-1} + \frac{3}{2}Y_n) \]
where \(n\) is assumed to be even for convenience.
\end{description}
\end{document}