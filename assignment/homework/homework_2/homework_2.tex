% Created 2017-03-21 Tue 11:25
% Intended LaTeX compiler: pdflatex
\documentclass[a4paper,11pt]{article}
\usepackage[utf8]{inputenc}
\usepackage[T1]{fontenc}
\usepackage{graphicx}
\usepackage{grffile}
\usepackage{longtable}
\usepackage{wrapfig}
\usepackage{rotating}
\usepackage[normalem]{ulem}
\usepackage{amsmath}
\usepackage{textcomp}
\usepackage{amssymb}
\usepackage{capt-of}
\usepackage{hyperref}
\usepackage[margin=1.2in]{geometry}
\usepackage{setspace}
\singlespacing
\usepackage{parskip}
\usepackage{amsthm}
\usepackage{mathtools}
\newcommand{\dx}{\mathrm{d}}
\newcommand{\var}{\mathrm{var}}
\newcommand{\cov}{\mathrm{cov}}
\newcommand{\corr}{\mathrm{corr}}
\newcommand{\pr}{\mathrm{Pr}}
\setcounter{secnumdepth}{0}
\date{Due on April 1\(^{\text{st}}\)}
\title{Homework Set 2}
\hypersetup{
 pdfauthor={},
 pdftitle={Homework Set 2},
 pdfkeywords={},
 pdfsubject={},
 pdfcreator={Emacs 25.1.1 (Org mode 9.0.3)},
 pdflang={English}}
\begin{document}

\maketitle
All questions are from the end-of-chapter exercises. The question
numbers refer to those in the book. I highly recommend you reading the
textbook and lecture notes before completing the homework
questions. When reading the textbook, please pay attention to the
sections on how to interpret the estimated coefficients.

\section*{Exercises}
\label{sec:orgc6220ca}

\begin{description}
\item[{4.2}] Suppose that a random sample of 200 twenty-year-old mean is
selected from a population and that these men's height and
weight are recorded. A regression of weight on height yields
\begin{equation*}
\widehat{Weight} = -99.41 + 3.94 \times Height, R^2 = 0.81, SER = 10.2
\end{equation*}
where \(Weight\) is measured in pounds and \(Height\) is measured
in inches.
\begin{description}
\item[{a.}] What is the regression's weight prediction for someone who
is 70 in. tall? 65 in. tall? 74 in. tall?
\item[{b.}] A man has a late growth spurt and grows 1.5 in. over the
course of a year. What is the regression's prediction for
the increase in this man's weight?
\item[{c.}] Suppose that instead of measuring weight and height in
pounds and inches these variables are measured in
centimeters and kilograms. What are the regression estimates
from this new centimeter-kilogram regression? (Give all
results, estimated coefficients, \emph{R\(^{\text{2}}\)}, and \emph{SER})
\end{description}
\end{description}

\vspace{0.5cm}

\begin{description}
\item[{4.3}] A regression of average weekly earning (AWE, measured in
dollars) on age (measured in years) using a random sample of
college-educated full-time worker aged 25-65 yields the
following:
\begin{equation*}
\widehat{AWE} = 696.7+ 9.6 \times Age, R^2 = 0.023, SER = 624.1
\end{equation*}
\begin{description}
\item[{a.}] Explain what the coefficient values 696.7 and 9.6 mean.
\item[{b.}] The standard error of the regression (\emph{SER}) is 624.1. What
are the units of measurement for the \emph{SER}? (Dollars? Years?
Or is \emph{SER} unit-free?)
\item[{c.}] The regression \emph{R\(^{\text{2}}\)} is 0.023. What are the units of
measurement for the \emph{R\(^{\text{2}}\)}? (Dollars? Years? Or is \emph{R\(^{\text{2}}\)} unit-free)
\item[{d.}] What is the regression's predicted earnings for a
25-year-old worker? A 45-year-old worker?
\item[{e.}] Will the regression give reliable predictions for a
99-year-old worker? Why or why not?
\item[{f.}] Give what you know about the distribution of earnings, do
you think it is plausible that the distribution of errors in
the regression is normal? (\emph{Hint}: Do you think that the
distribution is symmetric or skewed? What is the smallest
value of earnings, and is it consistent with a normal
distribution?)
\item[{g.}] The average age in this sample is 41.6 years. What is the
average value of \emph{AWE} in the sample? (\emph{Hint}: Review Key
Concept 4.2)
\end{description}
\end{description}

\vspace{0.5cm}

\begin{description}
\item[{4.5}] A professor decides to run an experiment to measure the
effect of time pressure on final exam scores. He gives each
of the 400 students in his course the same final exam, but
some students have 90 minutes to complete the exam while
others have 120 minutes. Each students is randomly assigned
one of the examination times based on the flip of a coin. Let
\(Y_i\) denote the number of points scored on the exam by the
i\(^{\text{th}}\) student \((0 \leq Y_i \leq 100)\), let \(X_i\) denote
the amount of time that the student has to complete the exam
\((X_i = 90 or 120)\), and consider the regression model \(Y_i =
         \beta_0 + \beta_1 X_i + u_i\).
\begin{description}
\item[{a.}] Explain what the term \(u_i\) represents. Why will different
students have different values of \(u_i\)?
\item[{b.}] Explain why \(E(u_i | X_i) = 0\) for this regression model.
\item[{c.}] Are the other assumptions in Key Concept 4.3 satisfied? Explain.
\item[{d.}] The estimated regression is \(\hat{Y}_i = 49 + 0.24 X_i\).
\begin{description}
\item[{i.}] Compute the estimated regression's prediction for the
average score of students given 90 minutes to complete the
exam. Repeat for 120 minutes and 150 minutes.
\item[{ii.}] Compute the estimated gain in score for a student who is
given an additional 10 minutes on the exam.
\end{description}
\end{description}
\end{description}

\vspace{0.5cm}

\begin{description}
\item[{4.10}] Suppose that \(Y_i = \beta_0 + \beta_1 X_i + u_i\), where
\((X_i, u_i)\) are i.i.d., and \(X_i\) is a Bernoulli random
variable with \(\pr(X=1) = 0.20\). When \(X=1, u_i \text{ is }
          N(0, 4)\); when \(X=0, u_i \text{ is } N(0, 1)\).
\begin{description}
\item[{a.}] Show that the regression assumptions in Key Concept 4.3 are satisfied.
\item[{b.}] Derive an expression for the large-sample variance of
\(\hat{\beta}_1\). (\emph{Hint}: Evaluate the terms in Equation
(4.21))
\end{description}
\end{description}

\vspace{0.5cm}

\begin{description}
\item[{4.12}] \begin{description}
\item[{a.}] Show that the regression \(R^2\) in the regression of \(Y\) on
\(X\) is the squared value of the sample correlation between
\(X\) and \(Y\). That is show that \(R^2 = r^2_{XY}\).
\item[{b.}] Show that the \(R^2\) from the regression of \(Y\) on \(X\) is the
same as the \(R^2\) from the regression of \(X\) on \(Y\).
\item[{c.}] Show that \(\hat{\beta}_1 = r_{XY}(s_Y/s_X)\), where \(r_{XY}\)
is the sample correlation between \(X\) and \(Y\), and \(s_Y\) and
\(s_Y\) are the sample standard deviations of \(X\) and \(Y\).
\end{description}
\end{description}

\section*{Empirical Exercise}
\label{sec:org2c1f4fb}

For the empirical exercise, you need to include the table for
regression results, the graphs, like the scatterplot, and the R or
STATA codes. The program codes should be appended at the end of all
answers.

\begin{description}
\item[{E4.2}] On the text Web site
\url{http:://www.pearsonhighered.com/stock\_watson/}, you will find
a data file \textbf{TeachingRatings} that contains data on course
evaluations, course characteristics, and professor
characteristics for 463 courses at the University of Texas
at Austin. A detailed description is given in
\textbf{TeachingRatings\_Description}, also available on the Web
site. One of the characteristics is an index of the
professor's "beauty" as rated by a panel of six judges. In
this exercise, you will investigate how course evaluations
are related to the professor's beauty.
\begin{description}
\item[{a.}] Construct a scatterplot of average course evaluations
(\emph{Course\_Eval}) on the professor's beauty (\emph{Beauty}). Does
there appear to be a relationship between the variables?
\item[{b.}] Run a regression of average course evaluations
(\emph{Course\_Eval}) on the professor's beauty (\emph{Beauty}). What
is the estimated intercept? What is the estimated slope?
Explain why the estimated intercept is equal to the sample
mean of \emph{Course\_Eval}. (\emph{Hint}: What is the sample mean of \emph{Beauty}?)
\item[{c.}] Professor Watson has an average value of \emph{Beauty}, while
Professor Stock's value of \emph{Beauty} is one standard
deviation above the average. Predict Professor Stock's and
Professor Watson's course evaluations.
\item[{d.}] Comment on the size of the regression's slope. Is the
estimated effect of \emph{Beauty} on \emph{Course\_Eval} large or
small? Explain what you mean by "large" and "small".
\item[{e.}] Does \emph{Beauty} explain a large fraction of the variance in
evaluations across courses? Explain.
\end{description}
\end{description}
\end{document}