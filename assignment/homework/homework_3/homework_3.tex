% Created 2017-04-10 Mon 16:28
% Intended LaTeX compiler: pdflatex
\documentclass[a4paper,11pt]{article}
\usepackage[utf8]{inputenc}
\usepackage[T1]{fontenc}
\usepackage{graphicx}
\usepackage{grffile}
\usepackage{longtable}
\usepackage{wrapfig}
\usepackage{rotating}
\usepackage[normalem]{ulem}
\usepackage{amsmath}
\usepackage{textcomp}
\usepackage{amssymb}
\usepackage{capt-of}
\usepackage{hyperref}
\usepackage[margin=1.2in]{geometry}
\usepackage{setspace}
\singlespacing
\usepackage{parskip}
\usepackage{amsthm}
\usepackage{mathtools}
\newcommand{\dx}{\mathrm{d}}
\newcommand{\var}{\mathrm{var}}
\newcommand{\cov}{\mathrm{cov}}
\newcommand{\corr}{\mathrm{corr}}
\newcommand{\pr}{\mathrm{Pr}}
\setcounter{secnumdepth}{0}
\date{Due on April 17\(^{\text{th}}\)}
\title{Homework Set 3}
\hypersetup{
 pdfauthor={},
 pdftitle={Homework Set 3},
 pdfkeywords={},
 pdfsubject={},
 pdfcreator={Emacs 25.1.1 (Org mode 9.0.3)},
 pdflang={English}}
\begin{document}

\maketitle
All questions are from the end-of-chapter exercises. The question
numbers refer to those in the book. I highly recommend you reading the
textbook and lecture notes before completing the homework
questions. When reading the textbook, please pay attention to the
sections on how to interpret the estimated coefficients.

\section*{Exercises}
\label{sec:org5f90c46}

\begin{description}
\item[{5.5}] In the 1980s Tennessee conducted an experiment in which
kindergarten students were randomly assigned to "regular" and
"small" classes, and given standardized tests at the end of
the year. (Regular classes contained approximately 24
students, and small classes contained approximately 15
students.) Suppose that, in the population, the standardized
tests have a mean score of 925 points and a standard
deviation of 75 points. Let \emph{SmallClass} denote a binary
variable equal to 1 if the student is assigned to a small
class and equal to 0 otherwise. A regression of \emph{TestScore}
on \emph{SmallClass} yields
\begin{equation*}
\widehat{TestScore} = \underset{\displaystyle (1.6)}{918.0} + \underset{\displaystyle (2.5)}{13.9} \times SmallClass,\, R^2 = 0.01,\, SER = 74.6.
\end{equation*}
\begin{description}
\item[{a.}] Do small classes improve test scores? By how much? Is the
effect large? Explain.
\item[{b.}] Is the estimated effect of class size on test scores
statistically significant? Carry out a test at the 5\% level.
\item[{c.}] Construct a 99\% confidence interval for the effect of
\emph{SmallClass} on test score.
\end{description}
\end{description}

\vspace{0.2cm}

\begin{description}
\item[{5.6}] Refer to the regression described in Exercise 5.5.
\begin{description}
\item[{a.}] Do you think that the regression errors plausibly are
homoskedastic? Explain.
\item[{b.}] \(SE(\hat{\beta}_1)\) was computed using Equation
(5.3). Suppose that the regression errors were
homoskedastic: Would this affect the validity of the
confidence interval constructed in Exercise 5.5(c)?
Explain.
\end{description}
\end{description}

\vspace{0.2cm}

\begin{description}
\item[{5.8}] Suppose that \((Y_i, X_i)\) satisfy the assumptions in Key
Concept 4.3 and, in addition, \(u_i \sim N(0, \sigma^2_u)\) and
is independent of \(X_i\). A sample of size \(n=30\) yields
\begin{equation*}
\hat{Y} = \underset{\displaystyle (10.2)}{43.2} + \underset{\displaystyle (7.4)}{61.5}X,\, R^2 = 0.54,\, SER = 1.52
\end{equation*}
where the numbers in parentheses are the homoskedastic-only
standard errors for the regression coefficients.
\begin{description}
\item[{a.}] Construct a 95\% confidence interval for \(\beta_0\).
\item[{b.}] Test \(H_0: \beta_1 = 55 \text{ vs. } H_1: \beta_1 \neq 55\)
at the 55\% level.
\item[{c.}] Test \(H_0: \beta_1 = 55 \text{ vs. } H_1: \beta_1 > 55\) at
the 5\% level.
(\emph{Hint}: this problem involves small samples with normally
distributed errors. So what distribution does the t-statistic
follow?)
\end{description}
\end{description}

\vspace{0.2cm}

\begin{description}
\item[{5.10}] Let \(X_i\) denote a binary variable and consider the
regression \(Y_i = \beta_0 + \beta_1 X_i + u_i\). Let
\(\bar{Y}_0\) denote the sample mean for observations with
\(X=0\) and \(\bar{Y}_1\) denote the sample mean for
observations with \(X=1\). Show that \(\hat{\beta}_0 =
          \bar{Y}_0,\, \hat{\beta}_0 + \hat{\beta}_1 = \bar{Y}_1,\,
          \text{ and } \hat{\beta}_1 = \bar{Y}_1 - \hat{Y}_0\).
\end{description}

\vspace{0.2cm}

\begin{description}
\item[{5.14}] Suppose that \(Y_i = \beta X_i + u_i\), where \((u_i, X_i)\)
satisfy the Gauss-Markov conditions given in Equation
(5.31).
\begin{description}
\item[{a.}] Derive the least squares estimator of \(\beta\) and show that
it is a linear function of \(Y_1, \ldots, Y_n\).
\item[{b.}] Show that the estimator is conditionally unbiased.
\item[{c.}] Derive the conditional variance of the estimator.
\item[{d.}] (Optional)\footnote{That this question is optional means you do not need to complete this
question. However, if you do, you will earn an extra point in your
homework.} Prove that the estimator is BLUE.
\end{description}
\end{description}

\section*{Empirical Exercise}
\label{sec:org264beb3}

For the empirical exercise, you need to explain your results and to
include the table for regression results, the graphs, like the
scatterplot, and the R or STATA codes. The program codes should be
appended at the end of all answers.

\begin{description}
\item[{E5.2}] Using the data set \texttt{TeachingRatings} described in Empirical
Exercise E4.2, run a regression of \texttt{Course\_Eval} on
\texttt{Beauty}. Is the estimated regression slope coefficient
statistically significant? That is, can you reject the null
hypothesis \(H_0: \beta_1 = 0\) versus a two-sided alternative
at the 10\%, 5\%, or 1\% significance level? What is the
p-value associated with coefficient's t-statistic?
\end{description}
\end{document}