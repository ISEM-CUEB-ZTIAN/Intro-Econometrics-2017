% Created 2017-04-19 Wed 09:22
% Intended LaTeX compiler: pdflatex
\documentclass[a4paper,11pt]{article}
\usepackage[utf8]{inputenc}
\usepackage[T1]{fontenc}
\usepackage{graphicx}
\usepackage{grffile}
\usepackage{longtable}
\usepackage{wrapfig}
\usepackage{rotating}
\usepackage[normalem]{ulem}
\usepackage{amsmath}
\usepackage{textcomp}
\usepackage{amssymb}
\usepackage{capt-of}
\usepackage{hyperref}
\usepackage[margin=1.2in]{geometry}
\usepackage{setspace}
\singlespacing
\usepackage{parskip}
\usepackage{amsthm}
\usepackage{mathtools}
\newcommand{\dx}{\mathrm{d}}
\newcommand{\var}{\mathrm{Var}}
\newcommand{\cov}{\mathrm{Cov}}
\newcommand{\corr}{\mathrm{corr}}
\newcommand{\pr}{\mathrm{Pr}}
\setcounter{secnumdepth}{0}
\date{Due on May 3rd, 2017}
\title{Homework Set 4}
\hypersetup{
 pdfauthor={},
 pdftitle={Homework Set 4},
 pdfkeywords={},
 pdfsubject={},
 pdfcreator={Emacs 25.1.1 (Org mode 9.0.3)}, 
 pdflang={English}}
\begin{document}

\maketitle
All questions are from the end-of-chapter exercises. The question
numbers refer to those in the book. I highly recommend you reading the
textbook and lecture notes before completing the homework
questions. When reading the textbook, please pay attention to the
sections on how to interpret the estimated coefficients.

\section*{Exercises}
\label{sec:orgc818fa3}
\begin{description}
\item[{6.5}] Data were collected from a random sample of 220 home sales
from a community in 2003. Let \emph{Price} denote the selling
price (in \$1000), \emph{BDR} denote the number of bedrooms, \emph{Bath}
denote the number of bathrooms, \emph{Hsize} denote the size of
the house (in square feet), \emph{Lsize} denote the lot size (in
square feet), \emph{Age} denote the age of the house (in years),
and \emph{Poor} denote a binary variable that is equal to 1 if the
condition of the house is reported as "poor". An estimated
regression yields
\begin{equation*}
\begin{split}
\widehat{Price} =& 119.2 + 0.485 BDR + 23.4 Bath + 0.156
Hsize + 0.002 Lsize \\
&+ 0.090 Age - 48.8 Poor,\, \bar{R}^2 = 0.72,\,
SER = 41.5
\end{split}
\end{equation*}
\begin{description}
\item[{a.}] Suppose that a home owner converts part of an existing family
room in the house into a new bath room. What is the expected
increase in the value of the house?
\item[{b.}] Suppose that a homeowner adds a new bathroom to her house,
which increases the size of the house by 100 square
feet. What is the expected increase in the value of the house?
\item[{c.}] What is the loss in value if a homeowner lets his house run
down so that its condition becomes "poor"?
\item[{d.}] Compute the \(R^2\) for the regression.
\end{description}

\item[{6.6}] A researcher plans to study the causal effect of police on
crime using data from a random sample of U.S. counties. He
plans to regress the county's crime rate on the (per capita)
size of the county's police force.
\begin{description}
\item[{a.}] Explain why this regression is likely to suffer from omitted
variable bias. Which variables would you add to the
regression to control for important omitted variables?
\item[{b.}] Use your answer to (a) and the expression for omitted
variable bias given in Equation (6.1) to determine whether
the regression will likely over- or underestimate the effect
of police on the crime rate. (That is, do you think that
\(\hat{\beta}_1 > \beta_1\) or \(\hat{\beta}_1 < \beta_1\)?)
\end{description}

\item[{6.10}] \((Y_i, X_{1i}, X_{2i})\) satisfy the assumptions in Key
Concept 6.4; in addition, \(\var(u_i | X_{1i}, X_{2i}) = 4\)
and \(\var(X_{1i}) = 6\). A random sample of size \(n=400\) is
drawn from the population.
\begin{description}
\item[{a.}] Assume that \(X_1\) and \(X_2\) are uncorrelated. Compute the
variance of \(\hat{\beta}_1\). (\emph{Hint}: Look at Equation
(6.17) in the Appendix 6.2)
\item[{b.}] Assume that \(\corr(X_1, X_2) = 0.5\). Compute the variance of
\(\hat{\beta}_1\).
\item[{c.}] Comment on the following statements: "When \(X_1\) and \(X_2\)
are correlated, the variance of \(\hat{\beta}_1\) is larger
than it would be if \(X_1\) and \(X_2\) were uncorrelated. Thus
if you are interested in \(\beta_1\), it is best to leave
\(X_2\) out of the
regression if it is correlated with \(X_1\)."
\end{description}

\item[{6.11}] (Require calculus) Consider the regression model
\[ Y_i = \beta_1 X_{1i} + \beta_2 X_{2i} + u_i \]
for \(i=1,\ldots,n\). (Notice that there is no constant term
in the regression.)
\begin{description}
\item[{a.}] Specify the least squares function that is minimized by OLS.
\item[{b.}] Compute the partial derivatives of the objective function
with respect to \(b_1\) and \(b_2\).
\item[{c.}] Suppose \(\sum_{i=1}^n X_{1i}X_{2i} = 0\). Show that
\(\hat{\beta}_1 = \sum_{i=1}^n X_{1i}Y_i/\sum_{i=1}^n
          X_{1i}^2\).
\item[{d.}] Suppose \(\sum_{i=1}^n X_{1i}X_{2i} \neq 0\). Derive an
expression for \(\hat{\beta}_1\) as a function of the data
\((Y_i, X_{1i}, X_{2i}),\, i=1,\ldots, n\).
\item[{e.}] Suppose that the model includes an intercept:
\(Y_i = \beta_0 + \beta_1 X_{1i} + \beta_2 X_{2i} +
          u_i\). Show that the least squares estimators satisfy
\(\hat{\beta}_0 = \bar{Y} - \hat{\beta}_1 \bar{X}_1 -
          \hat{\beta}_2 \bar{X}_2\).
\item[{f.}] As in (e), suppose that the model contains an
intercept. Also suppose that \(\sum_{i=1}^n (X_{1i} -
          \bar{X}_1)(X_{2i} - \bar{X}_2) = 0\). Show that
\(\hat{\beta}_1 = \sum_{i=1}^n (X_{1i} - \bar{X}_1)(Y_i -
          \bar{Y})/\sum_{i=1}^n (X_{1i} - \bar{X}_1)^2\). How does this
compare to the OLS estimator of \(\beta_1\) from the
regression that omit \(X_2\)?
\end{description}
\end{description}

\section*{Empirical Exercises}
\label{sec:orgfb1df83}
\begin{description}
\item[{E6.1}] Using the data set \textbf{TeachingRatings} described in Empirical
Exercise 4.2, carry out the following exercise.
\begin{description}
\item[{a.}] Run a regression of \texttt{Course\_Eval} on \texttt{Beauty}. What is the
estimated slope?
\item[{b.}] Run a regression of \texttt{Course\_Eval} on \texttt{Beauty}, including
some additional variables to control for the type of course
and professor characteristics. In particular, include as
additional regressors \texttt{Intro}, \texttt{OneCredit}, \texttt{Female},
\texttt{Minority}, and \texttt{NNEnglish}. What is the estimated effect of
\texttt{Beauty} on \texttt{Course\_Eval}? Does the regression in (a) suffer
from important omitted variable bias?
\item[{c.}] Estimate the coefficient on \texttt{Beauty} for the multiple
regression model in (b) using the three-step process in
Appendix 6.3 (the Frisch-Waugh theorem). Verify that the
three-step-process yields the same estimated coefficient for
\texttt{Beauty} as that obtained in (b).
\item[{d.}] Professor Smith is a black male with average beauty and is a
native English speaker. He teaches a three-credit
upper-division course. Predict Professor Smith's course
evaluation.
\end{description}
\end{description}
\end{document}