% Created 2017-05-14 Sun 21:58
% Intended LaTeX compiler: pdflatex
\documentclass[a4paper,11pt]{article}
\usepackage[utf8]{inputenc}
\usepackage[T1]{fontenc}
\usepackage{graphicx}
\usepackage{grffile}
\usepackage{longtable}
\usepackage{wrapfig}
\usepackage{rotating}
\usepackage[normalem]{ulem}
\usepackage{amsmath}
\usepackage{textcomp}
\usepackage{amssymb}
\usepackage{capt-of}
\usepackage{hyperref}
\usepackage[margin=1.2in]{geometry}
\usepackage{setspace}
\singlespacing
\usepackage{parskip}
\usepackage{amsthm}
\usepackage{mathtools}
\newcommand{\dx}{\mathrm{d}}
\newcommand{\var}{\mathrm{var}}
\newcommand{\cov}{\mathrm{cov}}
\newcommand{\corr}{\mathrm{corr}}
\newcommand{\pr}{\mathrm{Pr}}
\setcounter{secnumdepth}{0}
\date{Due on May 22th, 2017}
\title{Homework Set 5}
\hypersetup{
 pdfauthor={},
 pdftitle={Homework Set 5},
 pdfkeywords={},
 pdfsubject={},
 pdfcreator={Emacs 25.1.1 (Org mode 9.0.3)}, 
 pdflang={English}}
\begin{document}

\maketitle
All questions are from the end-of-chapter exercises in Chapter 7. The question
numbers refer to those in the book. I highly recommend you reading the
textbook and lecture notes before completing the homework
questions. When reading the textbook, please pay attention to the
sections on how to interpret the estimated coefficients.

\section*{Exercises}
\label{sec:org64a5d87}

\begin{description}
\item[{7.7}] Question 6.5 reported the following regression (where
standard erros have been added): 
\begin{equation*}
\begin{split}
\widehat{Price} &= \underset{(23.9)}{119.2} + \underset{(2.61)}{0.485}BDR + \underset{(8.94)}{23.4}Bath + \underset{(0.011)}{0.156} Hsize + \underset{(0.00048)}{0.002}Lsize \\
&+ \underset{(0.311)}{0.090} Age - \underset{(10.5)}{48.8}Poor,\, \bar{R}^2 = 0.72,\, SER = 41.5
\end{split}
\end{equation*}
\begin{description}
\item[{a.}] is the coefficient on \(BDR\) statistically significantly
different from zero?
\item[{b.}] Typically five-bedroom houses sell for much more than
two-bedroom houses. Is this consistent with your answer to
(a) and with the regression more generally?
\item[{c.}] A homeowner purchase 2000 square feet from an adjacent
lot. Construct a 99\% confident interval for the change in
the value of her house.
\item[{d.}] Lot size is measured in square feet. do you think that
another scale might be more appropriate? Why or why not?
\item[{e.}] The F-statistic fro omitting \(BDR\) and \(Age\) from the
regression is \(F = 0.08\). Are the coefficients on \(BDR\) and
\(Age\) statistically different from zero at the 10\% level?
\end{description}
\end{description}

\vspace{0.5cm}

\begin{description}
\item[{7.8}] Referring to Table 7.1 in the text:
\begin{description}
\item[{a}] Construct and compute the \(R^2\) from \(\bar{R}^2\) for each of
the regression.
\item[{b}] Construct the homoskedasticity-only F-statistic for testing
\(\beta_3 = \beta_4 = 0\) in the regression shown in column
(5). Is the statistic significant at the 5\% level?
\item[{c}] Construct a 99\% confidence interval for \(\beta_1\) for the
regression in column (5).
\end{description}
\end{description}

\vspace{0.5cm}

\begin{description}
\item[{7.9}] Consider the regression model \(Y_i = \beta_0 + \beta_1
         X_{1i} + \beta_2 X_{2i} + u_i\). Use Approach \#2 from Section
7.3 to transform the regression so that you can use a
t-statistic to test
\begin{description}
\item[{a.}] \(\beta_1 = \beta_2\)
\item[{b.}] \(\beta_1 + a\beta_2 = 0\), where \(a\) is a constant;
\item[{c.}] \(\beta_1 + \beta_2 = 1\). (\emph{Hint}: You must redefine the
dependent variable in the regression)
\end{description}
\end{description}

\vspace{0.5cm}

\begin{description}
\item[{7.11}] A school district undertakes an experiment to estimate the
effect of class size on test scores in second grade
classes. The district assigns 50\% of its previous year's
first graders to small second-grade classes (18 students per
classroom) and 50\% to regular-size classes (21 students per
classroom). Students new to the district are handled
differently: 20\% are randomly assigned to small classes and
80\% to regular-size classes. At the end of the second-grade
school year, each student is given a standardized exam. let
\(Y_i\) denote the exam score for i\(^{\text{th}}\) student, \(X_{1i}\)
denote a binary variable that equals 1 if the student is
assigned to small class, and \(X_{2i}\) denote a binary
variable that equals 1 if the student is newly enrolled. Let
\(\beta_1\) denote the class effect on test scores of reducing
class size from regular to small.
\begin{description}
\item[{a.}] Consider the regression \(Y_i = \beta_0 + \beta_1 X_{1i} +
          u_i\). Do you think that \(E(u_i | X_{1i}) = 0\)? Is the OLS
estimator of \(\beta_1\) unbiased and consistent? Explain.
\item[{b.}] Consider the regression \(Y_i = \beta_0 + \beta_1 X_{1i} +
          \beta_2 X_{2i} + u_i\). Do you think that \(E(u_i | X_{1i},
          X_{2i})\) depends on \(X_1\)? Is the OLS estimator of \(\beta_{\text{1}}\)
unbiased and consistent? Explain. Do you think that \(E(u_i |
          X_{1i}, X_{2i})\) depends on \(X_2\)? Will the OLS estimator of
\(\beta_{\text{2}}\) provide an unbiased and consistent estimate of the
causal effect of transferring to a new school (that is,
being a newly-enrolled student)? Explain.
\end{description}
\end{description}

\section*{Empirical Exercise}
\label{sec:orgc8f98de}
\begin{description}
\item[{E7.1}] use the data set \textbf{CPS08} described in Empirical Exercise 4.1
to answer the following questions.
\begin{description}
\item[{a.}] Run a regression of average hourly earnings (\emph{AHE}) on age
(\emph{Age}). What is the estimated intercept? What is the
estimated slope?
\item[{b.}] Run a regression of \emph{AHE} on \emph{Age}, gender (\emph{Female}), and
education (\emph{Bachelor}). What is the estimated effect of
\emph{Age} on earnings? Construct a 95\% confidence interval for
the coefficient on \emph{Age} in the regression.
\item[{c.}] Are the results from the regression in (b) substantively
different from the results in (a) regarding the effects of
\emph{Age} and \emph{AHE}? Does the regression in (a) seem to suffer
from omitted variable bias?
\item[{d.}] Bob is a 26-year-old male worker with a high school
diploma. Predict Bob's earnings using the estimated
regression in (b). Alexis is a 30-year-old female worker
with a college degree. Predict Alexis's earnings using the regression.
\item[{e.}] Compare the fit of the regression in (a) and (b) using the
regression standard errors, \(R^2\) and \(\bar{R}^2\). Why are
the \(R^2\) and \(\bar{R}^2\) so similar in regression (b)?
\item[{f.}] Are gender and education determinants of earnings? Test the
null hypothesis that \emph{Female} can be deleted from the
regression. Test the null hypothesis that \emph{Bachelor} can be
deleted from the regression. Test the null hypothesis that
both \emph{Female} and \emph{Bachelor} can be deleted from the
regression.
\item[{g.}] A regression will suffer from omitted variable bias when two
conditions hold. What are these two conditions? Do these
conditions seem to hold here?
\end{description}
\end{description}
\end{document}