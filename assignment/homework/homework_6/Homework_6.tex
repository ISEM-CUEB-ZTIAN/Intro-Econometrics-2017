% Created 2017-05-31 Wed 09:26
% -*- latex-run-command: pdflatex -*-
\documentclass[a4paper,11pt]{article}
\usepackage[utf8]{inputenc}
\usepackage[T1]{fontenc}
\usepackage{graphicx}
\usepackage{grffile}
\usepackage{longtable}
\usepackage{wrapfig}
\usepackage{rotating}
\usepackage[normalem]{ulem}
\usepackage{amsmath}
\usepackage{textcomp}
\usepackage{amssymb}
\usepackage{capt-of}
\usepackage{hyperref}
\usepackage[margin=1.2in]{geometry}
\usepackage{setspace}
\singlespacing
\usepackage{parskip}
\usepackage{amsthm}
\usepackage{mathtools}
\newcommand{\dx}{\mathrm{d}}
\newcommand{\var}{\mathrm{var}}
\newcommand{\cov}{\mathrm{cov}}
\newcommand{\corr}{\mathrm{corr}}
\newcommand{\pr}{\mathrm{Pr}}
\setcounter{secnumdepth}{0}
\date{Due on TBA}
\title{Homework Set 6}
\hypersetup{
 pdfauthor={},
 pdftitle={Homework Set 6},
 pdfkeywords={},
 pdfsubject={},
 pdfcreator={Emacs 25.1.1 (Org mode 9.0.7)}, 
 pdflang={English}}
\begin{document}

\maketitle
All questions are from the end-of-chapter exercises in Chapters 8
and 9. The question numbers refer to those in the book. I highly
recommend you reading the textbook and lecture notes before completing
the homework questions. When reading the textbook, please pay
attention to the sections on how to interpret the estimated
coefficients. Note that Exercise 9.12 is an optional question.

\section*{Exercises}
\label{sec:org251c7c7}
\begin{description}
\item[{8.3}] After reading this chapter's analysis of test scores and class
size, an educator comments, "In my experience, student
performance depends on class size, but not in the way your
regressions say. Rather, students do well when class size is
less than 20 students and do very poorly when class size is
greater than 25. There are no gains from reducing class size
below 20 students, the relationship is constant in the
intermediate region between 20 and 25 students, and there is
no loss to increasing class size when it is already greater
than 25." The educator is describing a "threshold effect" in
which performance is constant for class sizes less than 20,
then jumps and is constant for class sizes between 20 and 25,
and then jumps again for class sizes greater than 25. To
model these threshold effects, define the binary variables
\begin{gather*}
STRsmall = 1 \text{ if } STR < 20, \text{ and } STRsmall =
0 \text{ otherwise;} \\
STRmoderate = 1 \text{ if } 20 \leq STR \leq 25, \text{ and } STRmoderate = 0 \text{ otherwise; and } \\
STRlarge = 1 \text{ if } STR > 25, \text{ and } STRlarge = 0 \text{ otherwise}
\end{gather*}
\begin{description}
\item[{a.}] Consider the regression \(TestScore_i = \beta_0 + \beta_1
          STRsmall_i + \beta_2 STRlarge_i + u_i\). Sketch the regression
function relating \(TestScore\) to \(STR\) for hypothetical values
of the regression coefficients that are consistent with the
educator's statement.
\item[{b.}] A researcher tries to estimate the regression \(TestScore_i = \beta_0 + \beta_1
          STRsmall_i + \beta_2 STRmoderate_i + \beta_3 STRlarge + u_i\)
and finds that her computer crashes. Why?
\end{description}

\item[{8.7}] This problem is inspired by a study of the "gender gap" in
earnings in top corporate jobs [Bertrand and Hallock
(2001)]. The study compares total compensation among top
executives in a large set of U.S. public corporations in the
1990s. (Each year these publicly traded corporations must
report total compensation levels for their top five
executives.)
\begin{description}
\item[{a.}] Let \emph{Female} be an indicator variable that is equal to 1 for
females and 0 for males. A regression of the logarithm of
earnings onto \emph{Female} yields
\[ \widehat{\ln(Earnings)} = \underset{(0.01)}{6.48} -
          \underset{(0.05)}{-0.44}Female, SER = 2.65 \]
\begin{description}
\item[{i.}] The estimated coefficient on \emph{Female} is -0.44. Explain
what this value means.
\item[{ii.}] The \emph{SER} is 2.65. Explain what this value means.
\item[{iii.}] Does this regression suggest that female top executives
earn less than top male executives? Explain.
\item[{iv.}] Does this regression suggest that there is gender
discrimination? Explain.
\end{description}
\item[{b.}] Two new variables, the market value of the firm (a measure
of firm size, in millions of dollars) and stock return (a
measure of firm performance, in percentage points), are
added to the regression:
\begin{gather*}
\widehat{\ln(Earnings)} = \underset{\displaystyle (0.03)}{3.86} -
\underset{\displaystyle (0.04)}{0.28}Female +
\underset{\displaystyle (0.004)}{0.37}\ln(MarketValue)
+ \underset{\displaystyle (0.003)}{0.004}Return \\
n = 46,670, \bar{R}^2 = 0.345
\end{gather*}
\begin{description}
\item[{i.}] The coefficient on \(\ln(MarketValue)\) is 0.37. Explain
what this value means.
\item[{ii.}] The coefficient on \emph{Female} is now -0.28. Explain why
it has changed from the regression in (a).
\end{description}
\item[{c.}] Are large firms more likely to have female top executives
than small firms? Explain.
\end{description}

\item[{9.5}] The demand for a commodity is given by \(Q = \beta_0 + \beta_1
         P + u\), where \(Q\) denotes quantity, \(P\) denotes price, and
\(u\) denotes factors other than price that determine
demand. Supply for the commodity is given by \(Q = \gamma_0 +
         \gamma_1 P + v\), where \(v\) denotes factors other than price
that determine supply. Suppose that \(u\) and \(v\) both have a
mean of zero, have variances \(\sigma^2_u\) and \(\sigma^2_v\),
and are mutually uncorrelated.
\begin{description}
\item[{a.}] Solve the two simultaneous equations to show how \(Q\) and \(P\)
depend on \(u\) and \(v\).
\item[{b.}] Derive the means of \(P\) and \(Q\).
\item[{c.}] Derive the variance of \(P\), the variance of \(Q\), and the
covariance between \(Q\) and \(P\).
\item[{d.}] A random sample of observations of \((Q_i, P_i)\) is
collected, and \(Q_i\) is regressed on \(P_i\). (That is, \(Q_i\)
is the regressand and \(P_i\) is the regressor.) Suppose that
the sample is very large.
\begin{description}
\item[{i.}] Use your answers to (b) and (c) to derive values of the
regression coefficient (\emph{Hint}: Use Equation (4.7) and (4.8).)
\item[{ii.}] A researcher uses the slope of this regression as an
estimate of the slope of the demand function
(\(\beta_1\)). Is the estimated slope too large or too
small? (\emph{Hint}: Remember that demand curves slope down
and supply curves slope up.)
\end{description}
\end{description}

\item[{9.12 (Optional)}] Consider the one-variable regression model \(Y_i = \beta_0 +
          \beta_1 X_i + u_i\) and suppose that it satisfies the least
squares assumptions in Key Concept 4.3. The regressor \(X_i\)
is missing, but data on a related variable \(Z_i\) are
available, and the value of \(X_i\) is estimated using
\(\tilde{X}_i = E(X_i | Z_i)\). Let \(w_i = \tilde{X}_i -
          X_i\).
\begin{description}
\item[{a.}] Show that \(\tilde{X}_i\) is the minimum mean square error
estimator of \(X_i\) using \(Z_i\). That is, let \(\hat{X}_i =
          g(Z_i)\) be some other guess of \(X_i\) based on \(Z_i\) and show
that \(\var(\hat{X}_i - X_i) \geq \var(\tilde{X}_i -
          X_i)\). (\emph{Hint}: Review Exercise 2.27.)
\item[{b.}] Show that \(E(w_i | \tilde{X}_i) = 0\).
\item[{c.}] Suppose that \(E(u_i |Z_i) = 0\) and that \(\tilde{X}_i\) is
used as the regressor in place of \(X_i\). Show that
\(\hat{\beta}_1\) is consistent. Is \(\hat{\beta_0}\)
consistent?
\end{description}

\item[{9.13}] Assume that the regression model \(Y_i = \beta_0 + \beta_1
          X_i + u_i\) satisfies the least squares assumptions in Key
Concept 4.3. You and a friend collect a random sample of 300
observations on \(Y\) and \(X\).
\begin{description}
\item[{a.}] Your friend reports that he inadvertently scrambled the \(X\)
observations for 20\% of the sample. For these scrambled
observations, the value of \emph{X} does not correspond to \(X_i\)
for the i\(^{\text{th}}\) observation, but rather to the value of \(X\)
for some other observation. In the notation of Section 9.2,
the measured value of the regressor, \(\tilde{X}_i\), is equal
to \(X_i\), for 80\% of the observations, but is equal to a
randomly selected \(X_j\) for the remaining 20\% of the
observations. You regress \(Y_i\) on \(\tilde{X}_i\). Show that
\(E(\hat{\beta}_1) = 0.8\beta_1\).
\item[{b.}] Explain how you could construct an unbiased estimate of
\(\beta_1\) using the OLS estimator in (a).
\item[{c.}] Suppose now that your friend tells you that the \emph{X}'s were
scrambled for the first 60 observations, but that the
remaining 240 observations are correct. You estimate
\(\beta_1\) by regressing \emph{Y} on \emph{X} using only the correctly
measured 240 observations. Is this estimator of \(\beta_1\)
better than the estimator you proposed in (b)? Explain.
\end{description}
\end{description}

\section*{Empirical Exercise}
\label{sec:org9458d2e}
\begin{description}
\item[{E8.1}] Use the data set \textbf{CPS08} described in Empirical Exercise 4.1
to answer the following questions.
\begin{description}
\item[{a.}] Run a regression of average hourly earnings (\emph{AHE}) on age
(\emph{Age}), gender (\emph{Female}), and education (\emph{Bachelor}). If
\emph{Age} increases from 25 to 26, how are earnings expected to
change? If \emph{Age} increases from 33 to 34, how are earnings
expected to change?
\item[{b.}] Run a regression of the logarithm of average hourly
earnings, \(\ln(AHE)\), on \emph{Age}, \emph{Female}, and \emph{Bachelor}. If
\emph{Age} increases from 25 to 26, how are earnings expected to
change? If \emph{Age} increases from 33 to 34, how are earnings
expected to change?
\item[{c.}] Run a regression of the logarithm of average hourly
earnings, \(\ln(AHE)\), on \(\ln(Age)\), \emph{Female}, and
\emph{Bachelor}. If \emph{Age} increases from 25 to 26, how are
earnings expected to change? If \emph{Age} increases from 33 to
34, how are earnings expected to change?
\item[{d.}] Run a regression of the logarithm of average hourly
earnings, \(\ln(AHE)\), on \emph{Age}, \(Age^2\), \emph{Female}, and
\emph{Bachelor}. If \emph{Age} increases from 25 to 26, how are
earnings expected to change? If \emph{Age} increases from 33 to
34, how are earnings expected to change?
\item[{e.}] Do you prefer the regression in (c) to the regression in
(b)? Explain.
\item[{f.}] Do you prefer the regression in (d) to the regression in
(b)? Explain.
\item[{g.}] Do you prefer the regression in (d) to the regression in
(c)? Explain.
\item[{h.}] Pot the regression relation between \emph{Age} and \(\ln(AHE)\)
from (b), (c), and (d) for males with a high school
diploma. Describe the similarities and differences between
the estimated regression functions. Would your answer change
if you plotted the regression function for females with
college degree?
\item[{i.}] Run a regression of \(\ln(AHE)\) on \emph{Age}, \(Age^2\), \emph{Female},
\emph{Bachelor} and the interaction term \(Female \times
          Bachelor\). What does the coefficient on the interaction term
measure? Alexis is a 30-year-old female with a bachelor's
degree. What does the regression predict for her value of
\(\ln(AHE)\)? Jane is a 30-year-old female with a high school
degree. What does the regression predict for her value of
\(\ln(AHE)\)? What is the predicted difference between
Alexis's and Jane's earnings? Bob is a 30-year-old male with
a bachelor's degree. What does the regression predict for
his value of \(\ln(AHE)\)? Jim is a 30-year-old male with a
high school degree. What does the regression predict for his
value of \(\ln(AHE)\)? What is the predicted difference
between Bob's and Jim's earnings?
\item[{j.}] Is the effect of \emph{Age} on earnings different for men than
for women? Specify and estimate a regression that you can
use to answer this question?
\item[{k.}] Is the effect of \(Age\) on earnings different for high school
graduates than for college graduates? Specify and estimate a
regression that you can use to answer this question.
\item[{l.}] After running all these regressions (and any others that you
want to run), summarize the effect of age on earnings for
young workers.
\end{description}
\end{description}
\end{document}