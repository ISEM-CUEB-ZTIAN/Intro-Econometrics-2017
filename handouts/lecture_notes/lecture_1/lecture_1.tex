% Created 2017-02-10 Fri 10:15
% Intended LaTeX compiler: pdflatex
\documentclass[a4paper,11pt]{article}
\usepackage[utf8]{inputenc}
\usepackage[T1]{fontenc}
\usepackage{graphicx}
\usepackage{grffile}
\usepackage{longtable}
\usepackage{wrapfig}
\usepackage{rotating}
\usepackage[normalem]{ulem}
\usepackage{amsmath}
\usepackage{textcomp}
\usepackage{amssymb}
\usepackage{capt-of}
\usepackage{hyperref}
\usepackage[margin=1.2in]{geometry}
\usepackage{setspace}
\onehalfspacing
\usepackage{parskip}
\usepackage{amsthm}
\usepackage{amsmath}
\usepackage{mathtools}
\usepackage{hyperref}
\usepackage{graphicx}
\usepackage{tabularx}
\usepackage{booktabs}
\hypersetup{colorlinks,citecolor=black,filecolor=black,linkcolor=black,urlcolor=black}
\newtheorem{definition}{Definition}
\newtheorem{theorem}{Theorem}
\newcommand{\dx}{\mathrm{d}}
\newcommand{\var}{\mathrm{Var}}
\newcommand{\cov}{\mathrm{Cov}}
\newcommand{\corr}{\mathrm{Corr}}
\newcommand{\pr}{\mathrm{Pr}}
\newcommand{\rarrowd}[1]{\xrightarrow{\text{ \textit #1 }}}
\DeclareMathOperator*{\plim}{plim}
\newcommand{\plimn}{\plim_{n \rightarrow \infty}}
\setcounter{secnumdepth}{2}
\author{Zheng Tian}
\date{}
\title{Lecture 1: Economic Questions and Data}
\hypersetup{
 pdfauthor={Zheng Tian},
 pdftitle={Lecture 1: Economic Questions and Data},
 pdfkeywords={},
 pdfsubject={},
 pdfcreator={Emacs 25.1.1 (Org mode 9.0.3)}, 
 pdflang={English}}
\begin{document}

\maketitle

\section{What is Econometrics about?}
\label{sec:orga4c3199}

\subsection{Definition of Econometrics}
\label{sec:org142ff02}

Econometricians may give different definitions of Econometrics from
their own perspective.

Stock and Watson (2015) define Econometrics as
\begin{quote}
At a broad level, econometrics is the science and art of using
economic theory and statistical techniques to analyze economic
data.
\end{quote}


\subsection{The objective of Econometrics}
\label{sec:orge6d008f}

Frisch (1933) set the objectives of Econometrics as follows,
\begin{quote}
(Econometrics's) main objective shall be to promote studies that aim at
a unification of the theoretical-quantitative and the
empirical-quantitative approach to economic problems and that are
penetrated by constructive and rigorous thinking similar to that which
has come to dominate the natural sciences. \ldots{}\ldots{} Experience has shown
that each of these three viewpoints, that of statistics, economic
theory, and mathematics, is a necessary, but not by itself a
sufficient, condition for a real understanding of the quantitative
relations in modern economic life. It is the unification of all
three that is powerful. And it is this unification that constitutes
econometrics.
\end{quote}


\section{Economic Questions We Examine}
\label{sec:orgcdf6e69}

\subsection{Four practical questions}
\label{sec:org13400d8}

\begin{description}
\item[{Question 1}] does reducing class size improve elementary school education?

\item[{Question 2}] is there racial discrimination in the market for home loan?

\item[{Question 3}] how much do cigarette taxes reduce smoking?

\item[{Question 4}] what will the rate of inflation be next year?
\end{description}


\subsection{How does an econometrician formulate such questions?}
\label{sec:orgd61830c}

Ideally, an econometrician would design his/her econometric modeling
by the following steps

\begin{enumerate}
\item Establish a theoretical model to qualitatively describe how X could
cause Y, holding other factors constant. From the theoretical
model, put forward some hypotheses to validate the theory.
\item Find actual data to measure the variables in the theory
\item Set up an empirical model to test the theoretical model, using
available data.
\item Choose a suitable estimation method to estimate the empirical model.
\item Perform some hypothesis tests and model specification test to
validate the estimation results.
\item Based on the estimation and test results, determine whether the theory
is internally and externally valid.
\end{enumerate}


\section{Causal Effects and Idealized Experiments}
\label{sec:org4880463}

The success of an econometric analysis relies on whether the causal
effects between X and Y can be accurately identified, excluding the
influences of other factors.

\subsection{Randomized controlled experiment}
\label{sec:org1bdc749}

\subsubsection*{Controlled experiment}
\label{sec:org12efdd8}

Control group (no treatment) versus treatment group (with treatment)

\subsubsection*{Randomized experiment}
\label{sec:org51b68ff}
the treamtment is assigned randomly

\subsubsection*{Advantages and disadvantages}
\label{sec:org656204c}

\begin{description}
\item[{Advantages}] eliminate the possibility of a systematic relationship that could
blur the causal effects of the treatment

\item[{Disadvantages}] it is difficult to implement, especially for social
science
\end{description}


\section{Data Sources and Types}
\label{sec:orgf4364ca}
\subsection{{\bfseries\sffamily TODO} Experimental versus observational data}
\label{sec:org04c3a96}

\subsection{Cross-sectional data}
\label{sec:org21cdc25}

\begin{itemize}
\item heights of all 30 students in a class

\item total population of each province in China in 2014
\end{itemize}

\subsection{Time series data}
\label{sec:org90e2642}

\begin{itemize}
\item stock price of Company A by hour over the last month

\item consumer price index of China by month from 1990 to 2014
\end{itemize}

\subsection{Panel data}
\label{sec:org686247e}

\begin{itemize}
\item annual wage of a fixed group of respondents in a survey conducted by
a statistic agency in 1990, 1995, 2000, 2005, and 2010

\item GDP per capita of each country in Asia from 1990 to 2014
\end{itemize}
\end{document}