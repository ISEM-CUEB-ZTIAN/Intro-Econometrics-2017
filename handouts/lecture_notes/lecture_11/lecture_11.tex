% Created 2017-05-26 Fri 16:11
% Intended LaTeX compiler: pdflatex
\documentclass[a4paper,11pt]{article}
\usepackage[utf8]{inputenc}
\usepackage[T1]{fontenc}
\usepackage{graphicx}
\usepackage{grffile}
\usepackage{longtable}
\usepackage{wrapfig}
\usepackage{rotating}
\usepackage[normalem]{ulem}
\usepackage{amsmath}
\usepackage{textcomp}
\usepackage{amssymb}
\usepackage{capt-of}
\usepackage{hyperref}
\usepackage[margin=1.2in]{geometry}
\usepackage{setspace}
\onehalfspacing
\usepackage{parskip}
\usepackage{amsthm}
\usepackage{amsmath}
\usepackage{mathtools}
\usepackage{hyperref}
\usepackage{graphicx}
\usepackage{tabularx}
\usepackage{booktabs}
\usepackage{color}
\usepackage{caption}
\usepackage{subcaption}
\hypersetup{colorlinks,citecolor=black,filecolor=black,linkcolor=black,urlcolor=black}
\newtheorem{mydef}{Definition}
\newtheorem{mythm}{Theorem}
\newcommand{\dx}{\mathrm{d}}
\newcommand{\var}{\mathrm{Var}}
\newcommand{\cov}{\mathrm{Cov}}
\newcommand{\corr}{\mathrm{Corr}}
\newcommand{\pr}{\mathrm{Pr}}
\newcommand{\rarrowd}[1]{\xrightarrow{\text{ \textit #1 }}}
\DeclareMathOperator*{\plim}{plim}
\newcommand{\plimn}{\plim_{n \rightarrow \infty}}
\setcounter{secnumdepth}{2}
\author{Zheng Tian}
\date{}
\title{Lecture 11: Assessing Studies Based on Multiple Regression}
\hypersetup{
 pdfauthor={Zheng Tian},
 pdftitle={Lecture 11: Assessing Studies Based on Multiple Regression},
 pdfkeywords={},
 pdfsubject={},
 pdfcreator={Emacs 25.1.1 (Org mode 9.0.3)}, 
 pdflang={English}}
\begin{document}

\maketitle

\section{Introduction}
\label{sec:orgcaeaad6}
\subsection{Overview}
\label{sec:org91b551e}
The preceding lectures explain how to use multiple regression to
analyze the relationship among variables. In this lecture, we step
back and ask, What makes a study that uses multiple regression
reliable? We answer this question by assessing regression analysis
under the framework of internal and external validity. 

\subsection{Reading materials}
\label{sec:org8bb5631}
\begin{itemize}
\item Chapter 9 in \emph{Introduction to Econometrics} by Stock and Watson.
\end{itemize}

\section{Internal and External Validity}
\label{sec:orge8f148b}
The concepts of internal and external validity provide a general
framework for assessing whether a statistical or econometric study is
useful for answering a specific question of interest. We focus on
regression analysis that have the objective of estimating the causal
effect of a change in some independent variable on a dependent
variable. 

\subsection{The population and setting studied versus the population and setting of interest}
\label{sec:org1defef8}
\subsubsection*{The population and setting studied}
\label{sec:org27deca3}
\begin{itemize}
\item The population studied is the population of entities-people,
companies, school districts, and so forth-from which the sample is
drawn.
\item The setting studied refers to as the institutional, legal, social,
and economic environment in which the population studied fits in and
the sample is drawn.
\end{itemize}
\subsubsection*{The population and setting of interest}
\label{sec:org2896b51}
By contrast, the population and setting of interest is the population
and setting of entities to which the causal inferences from the study
are to be applied. 

\subsection{Definition of internal and external validity}
\label{sec:org818b2b6}
\begin{itemize}
\item \textbf{Internal validity}: the statistical inferences about causal
effects are valid for the population being studied.
\item \textbf{External validity}: the statistical inferences can be generalized from
the population and setting studied to other populations and
settings.
\end{itemize}

\subsection{Threats to internal validity}
\label{sec:org62aa4cd}
\subsubsection*{Internal validity consists of two components}
\label{sec:org44bcb35}

\begin{itemize}
\item The estimator of the causal effect should be unbiased and
consistent.
\item Hypothesis tests should have the desired significance level (the
actual rejection rate of the test under the null hypothesis should
equal its desired significance level), and the confidence intervals
should have the desired confidence level.
\end{itemize}
\subsubsection*{Internal validity in regression analysis}
\label{sec:org29249c9}

For a regression analysis of casual effects based on the OLS
estimation, the requirements for internal validity are that 
\begin{enumerate}
\item the OLS estimator is unbiased and consistent, and
\item the standard errors are computed in a way that makes confidence
intervals have the desired confidence level.
\end{enumerate}

\subsection{Threats to external validity}
\label{sec:org1b4f979}
Potential threats to external validity arise from differences between
the population and setting studied and the population and setting of
interest. 

\subsubsection*{Differences in populations}
\label{sec:org7047dae}

The causal effect might not be the same in the population studied and
the population of interest due to their differences in
\begin{itemize}
\item demographic and personal characteristics,
\item geographic and climate features, and
\item timing.
\end{itemize}

\subsubsection*{Differences in settings}
\label{sec:orga9df22b}
\begin{itemize}
\item Difference in institutional environment, laws, or physical
environment.
\end{itemize}

\subsubsection*{How to assess the external validity of a study}
\label{sec:orged7c541}
\begin{itemize}
\item External validity must be judged using specific knowledge of the
population and settings studied and those of interest.
\item We can compare two or more studies on different but related
populations. Formally, this comparison can be conducted using a
meta-analysis.
\end{itemize}

\section{Threats to Internal Validity of Multiple Regression Analysis}
\label{sec:org4d425d9}
We introduce five threats to the internal validity of regression studies:
\begin{enumerate}
\item Omitted variable bias
\item Wrong functional form
\item Errors-in-variables bias
\item Sample selection bias
\item Simultaneous causality bias
\end{enumerate}

All of these imply that \(E(u_i|X_{1i},…,X_{ki}) \neq 0\) so as to make
the OLS estimators biased and inconsistent.  

\subsection{Omitted variable bias}
\label{sec:org22aaf20}
Recall that omitted variable bias arises when a variable that both
determines \(Y\) and is correlated with one or more of the included
regressors is omitted from the regression. 

\subsubsection*{Solutions to omitted variable bias when the variable is observed or there are adequate control variables}
\label{sec:org234ee51}
\begin{itemize}
\item A trade-off between omitted variable bias and the precision of estimators
\label{sec:orgfe05d91}
\begin{itemize}
\item If you have the data on the omitted variable, or you have the data
on one or more control variables for an unobserved omitted variable,
we can add these additional regressors to avoid the violation of the
first least squares assumption, \(E(u | X ) = 0\) or to let the
conditional mean independence assumption hold, i.e., \(E(u|X, W) =
  E(u|X)\), so that the coefficient on the variable of interest is
unbiased and consistent.

\item Adding an additional independent variable may reduce the precision of the
estimators of the coefficients when the new variable actually does
not belong to the population regression function (i.e., its
population regression coefficient is zero), or when the new variable
is correlated with other regressors, resulting in imperfect
multicollinearity. 

Question: Why may adding an irrelevant variable reduce the precision of other
coefficients? (\emph{Hint: What does the Gauss-Markov Theorem indicate as
for the variance of the OLS estimators?})
\end{itemize}

\item Some guidelines to decide whether to include an additional variable
\label{sec:orga6da27a}

\begin{enumerate}
\item Identify the key coefficient(s) of interest.
\begin{itemize}
\item e.g., the student-teacher ratio in the test score regression.
\end{itemize}
\item \emph{a priori} reasoning
\begin{itemize}
\item What are the most likely sources of important omitted variable?
\item Answer the question using economic theory and expert knowledge.
\item Done before analyzing data.
\item Result in a base specification and a list of additional
questionable variables that might help mitigate possible omitted
variable bias.
\end{itemize}
\item Augment your base specification with the additional questionable
control variables.
\begin{itemize}
\item If the coefficients on control variables are statistically
significant or if the estimated coefficients of interest change
appreciably when control variables are included, then you should
consider modifying the base specification.
\item If not, exclude these control variables from the regression.
\end{itemize}
\item Present an accurate summary of your results in tabular form.
\begin{itemize}
\item This provides "full disclosure" to skeptical readers who can draw
their conclusions.
\end{itemize}
\end{enumerate}
\end{itemize}
\subsubsection*{Solutions to omitted variable bias when adequate control variables are not available}
\label{sec:org4ed7abe}
Adding an omitted variable is not an option if you do not have data on
that variable and if there are no adequate control variables. We
introduce three ways to circumvent omitted variable bias. 

\begin{itemize}
\item \textbf{Panel data}
\label{sec:org07fc9d1}

Panel data (or longitudinal data) consist of observations on the same \(n\) entities at two or
more time periods. If the data set contains observations on the
variables \(X\) and \(Y\), then the data are denoted
\[ (X_{it}, Y_{it}),\; i = 1, \ldots, n \text{ and } t = 1, \ldots, T \]
where the first subscript, \(i\), refers to the entity being observed
and the second subscript, \(t\), refers to the date at which it is
observed. 

The key of using panel data regression to circumvent omitted variable
bias lies in the idea that omitted variables that represent personal
characteristics do not change over time so that any changes in \(Y\)
over time cannot be caused by the omitted variable.

Suppose we have \(n\) entities and \(T\) observations for each
entity. \(X_{it}\) is the observed regressor, \(Y_{it}\) is the dependent
variable, and \(Z_i\) is the unobserved time-invariant variable
representing idiosyncratic characteristics of entity \(i\). We can set
up a linear regression model as follows 
\[ Y_{it} = \beta_0 + \beta_1 X_{it} +
\beta_2 Z_i + u_{it} \] 
This model is a simple representation of the \textbf{fixed effects} panel
data regression model, in which \(Z_i\) is usually defined as a dummy
variable for entity \(i\). 

\item \textbf{Instrumental variable}
\label{sec:org402643d}

If the omitted variable(s) cannot be measured, we can use an instrumental
variables (IV) regression. Suppose that in the simple linear
regression model
\[ Y_i = \beta_0 + \beta_1 X_i + u_i, i = 1, \ldots, n \]
\(X_i\) and \(u_i\) are correlated due to unobserved omitted
variables. Then we can use an instrumental variable \(Z\) to account for
the part in \(X\) that is correlated with \(u\). 

For an instrumental variable \(Z\) to be valid, it
must satisfy two conditions: 
\begin{enumerate}
\item \textbf{Instrument relevance}: \(\corr(Z_i, X_i) \neq 0\)
\item \textbf{Instrument exogeneity}: \(\corr(Z_i, u_i) = 0\)
\end{enumerate}

The model is estimated using the Two-Stage-Least-Squares (TSLS) method
which basically consists of two steps:
\begin{description}
\item[{Stage 1}] Regress \(X_i\) on \(Z_i\), including an intercept, obtain
the predicted values, \(\hat{X}_i\).
\item[{Stage 2}] Regress \(Y_i\) on \(\hat{X}_i\), including an intercept; the
coefficient on \(\hat{X}_i\) is the TSLS estimator
\(\hat{\beta}_1^{TSLS}\).
\end{description}

\item \textbf{Randomized controlled experiment}
\label{sec:org4205210}

The third solution is to use a research design in which the effect of
interest is studied using a randomized controlled
experiment. Randomized controlled experiments are discussed in
Chapter 12.
\end{itemize}

\subsection{Misspecification of the functional form of the regression function}
\label{sec:org2609c79}
\begin{itemize}
\item Functional form misspecification arises when the functional form of
the estimated regression function differs from the functional form of
the population regression function. 
\begin{itemize}
\item e.g., nonlinear vs. linear models
\end{itemize}
\item Functional form misspecification bias can be considered as a type of
omitted variable bias, in which the omitted variables are the terms
that reflect the missing nonlinear aspects of the regression
function. 
\begin{itemize}
\item e.g., missing the quadratic term
\end{itemize}
\end{itemize}

\subsubsection*{Solutions to functional form misspecification}
\label{sec:org0084eb0}
\begin{itemize}
\item Plotting the data and the estimated regression function.
\item Use a different functional form.
\begin{itemize}
\item Continuous dependent variable:  use the “appropriate” nonlinear
specifications in X (logarithms, interactions, etc.)
\item Discrete (example: binary) dependent variable:  need an extension of
multiple regression methods (“probit” or “logit” analysis for binary
dependent variables)
\end{itemize}
\end{itemize}

\subsection{Measurement error and errors-in-variable bias}
\label{sec:org15958b9}
Measurement errors often happen in practice. They may come from
respondents misstated answers to survey questions, from typographical
errors when data were entered into the database for the first time,
and from the malfunctions of machines when recording data. 

Measurement errors can occur in independent variables as well as the
dependent variable, of which their effects on the estimated
coefficients depend on the nature of the errors. Let's first focus on
errors in independent variable, which cause biased estimated
coefficients, referred to as \textbf{errors-in-variable bias}.

\subsubsection*{Definition of errors-in-variable bias}
\label{sec:orgf5af312}
Errors-in-variables bias in the OLS estimator arises when an
independent variable is measured imprecisely. This bias depends on the
nature of the measurement error and persists even if the sample size
is large.

\subsubsection*{Mathematical illustration}
\label{sec:org6e45f34}

Suppose a regressor \(X_i\) is imprecisely measured by
\(\tilde{X}_i\). That means that we observe \(\tilde{X}_i\) and use it in
estimation. 

Then consider a simple regression model 
\[ Y_i = \beta_0 + \beta_1 X_i + u_i  \]
in which \(E(u_i | X_i) = 0\) is satisfied. 

Since we use \(\tilde{X}_i\) other than \(X_i\) in estimation, we
rewrite the model in terms of \(\tilde{X}_i\), that is,
\begin{equation}
\begin{split}
Y_i &= \beta_0 + \beta_1 \tilde{X}_i + [\beta_1 (X_i - \tilde{X}_i) + u_i] \\
    &= \beta_0 + \beta_1 \tilde{X}_i + v_i \label{eq:err-in-var}
\end{split}
\end{equation}
where \(v_i = \beta_1(X_i - \tilde{X}_i) + u_i\) in which we define the
measurement error as \(w_i = \tilde{X}_i - X_i\), and assume \(E(w_i) =
0\) and \(\var(w_i) = \sigma^2_w\). 

If the measurement errors \(w_i\) is correlated with \(\tilde{X}_i\), then
the regressor \(\tilde{X}_i\) is correlated with the new error term
\(v_i\) and \(\hat{\beta}_i\) will be biased and inconsistent
The OLS estimator \(\hat{\beta}_1\) is biased since \(E(v_i |
\tilde{X}_i) \neq 0\). 

The precise size and direction of the bias in \(\hat{\beta}_1\) depend
on the correlation between \(\tilde{X}_i\) and the measurement error
\(w_i\). This correlation depends, in turn, on the specific nature of
the measurement error. 

\subsubsection*{The classical measurement error model}
\label{sec:org45d486a}
The classical measurement error model assumes that the errors are
purely random so that we assume \(\corr(w_i, X_i) = 0\) and \(\corr(w_i,
u_i) = 0\), but the errors are correlated with \(\tilde{X}_i\), that is,
\(\corr(\tilde{X}_i, w_i) \neq 0\). Then, we can prove that in this
model, the OLS estimator \(\hat{\beta}_1\) of Equation
(\ref{eq:err-in-var}) is inconsistent, and its the probability limit
is
\begin{equation}
\label{eq:eiv-lim}
\hat{\beta}_1 \rarrowd{p} \frac{\sigma^2_X}{\sigma^2_X + \sigma^2_w}\beta_1
\end{equation}

Since \(\frac{\sigma^2_X}{\sigma^2_X + \sigma^2_w} < 1\), Equation
(\ref{eq:eiv-lim}) implies that
\(\hat{\beta}_1\) is biased toward 0.
\begin{itemize}
\item When \(\sigma^2_w\) is very large, then \(\hat{\beta}_1 \rarrowd{p} 0\);
\item When \(\sigma^2_w\) is very small, then \(\hat{\beta}_1 \rarrowd{p} \beta_1\).
\end{itemize}

\begin{proof}
Since $\tilde{X}_i = X_i + w_i$, we have $\var(\tilde{X}_i) = \sigma^2_{X} + \sigma^2_w$. 

According to Equation (\ref{eq:eiv-lim}) and $\cov(X_i, u_i) = 0$, we have
\begin{gather*}
v_i = \beta_1 (X_i - \tilde{X}_i) + u_i = -\beta_1 w_i + u_i \\
\cov(\tilde{X}_i, w_i) = \cov(X_i + w_i, w_i) = \sigma^2_w \\
\cov(\tilde{X}_i, v_i) = -\beta_1 \cov(\tilde{X}_i, w_i) + \cov(\tilde{X}_i, u_i) = -\beta_1 \sigma^2_w
\end{gather*}

Recall that in Chapter 6 for a simple regression model, when the error term is correlated with the regressor,
like $\cov(\tilde{X}_i, v_i) \neq 0$, then $\hat{\beta_1}$ has the probability limit
\[\hat{\beta}_1 \rarrowd{p} \beta_1 + \frac{\cov(\tilde{X}_i, v_i)}{\var(\tilde{X}_i)} \]
for which the probability limit is just
\[ \beta_1 - \beta_1 \frac{\sigma^2_w}{\sigma^2_{\tilde{X}_i}} = \frac{\sigma^2_X}{\sigma^2_X + \sigma^2_w}\beta_1 \]
\end{proof}

\subsubsection*{Measurement error in Y}
\label{sec:orgf6206d3}
The effect of measurement error in Y is different from that in
X. Generally, measurement in Y that has conditional mean zero given
the regressors will not induce bias in the OLS coefficients. 

\begin{itemize}
\item Suppose Y has the classical measurement error, that is, what we
observe, \(\tilde{Y}_i\), is the true value of \(Y_i\) plus a purely
random error \(w_i\). Then, the regression model is 
\[ \tilde{Y}_i = \beta_0 + \beta_1 + v_i, \text{ where } v_i = w_i +
  u_i\]
\item If \(w_i\) and \(X_i\) are independently distributed so that \(E(w_i | X_i)
  = 0\), in which case \(E(v_i | X_i) = 0\), so \(\hat{\beta}_1\) is
unbiased.
\item Since \(\var(v_i) = \var(w_i) + \var(u_i) > \var(u_i)\), the variance
of \(\hat{\beta}_1\) is larger than it would be without measurement
error.
\end{itemize}

\subsubsection*{Solutions to errors-in-variable bias}
\label{sec:org8ee290d}
\begin{itemize}
\item Get an accurate measure of \(X\) as possible as you can.
\item Use an instrumental variable that is correlated with the actual
value of \(X_i\) but is uncorrelated with the measurement error.
\item Develop a mathematical model of the measurement error and use the
resulting formula to adjust the estimates. This requires specific
knowledge of the errors.
\end{itemize}

\subsection{Missing data and sample selection}
\label{sec:org1b8bb82}
Missing data are a common feature of economic data sets. Whether
missing data pose a threat to internal validity depends on why the
data are missing. We consider three cases of missing data. 

\subsubsection*{Missing data at random}
\label{sec:orgd0b776b}

When data are missing completely at random, unrelated with \(X\) and
\(Y\), then the effect is to reduce the sample size but not introduce
any estimation bias. 

\subsubsection*{Missing data based on \(X\)}
\label{sec:org3cb4231}

When the data are missing based on the value of a regressor but
unrelated with generating \(Y\), the effect is also to reduce the sample
size but not introduce bias. For example, we repeat an experiment
examining the influence of X on Y on several days and save the results
at different time. Suppose that time is a regressor, and we miss the
all data from 1 pm to 2 pm. If the missing data do not affect the
process of doing the experiment, then the estimate of the causal
effect of X on Y will still be unbiased.

\subsubsection*{Sample selection bias}
\label{sec:org0c259b9}

When the data are missing because of a selection process that is
related with the value of the dependent variable \(Y\), beyond depending
on the regressors \(X\), then this selection process can introduce
correlation between the error term and the regressors, resulting in
\textbf{sample selection bias}.

The sample selection problem can be cast either as a consequence of
nonrandom sampling or as a missing data problem, illustrated using the
following two examples. 

\begin{itemize}
\item Nonrandom sampling: Height of undergraduates 

The professor of Statistics asks you to estimate the mean height of
undergraduate males. You collect your data (obtain your sample) by
standing outside the basketball team’s locker room and recording the
height of the undergraduates who enter.
\begin{itemize}
\item Is this a good research design – will it yield an unbiased
estimate of undergraduate height?
\item You have sampled individuals in a way that was related to
the outcome Y (height), resulting in bias.
\end{itemize}

\item Missing data: Trade volume of pairs of countries

\begin{itemize}
\item The amount of commodities that two countries can trade depends on
GDP of two countries, industrial structures, factor abundance,
etc.
\item We can get the data on trade volume between pairs of countries
from World Bank, Penn World Table, etc.
\item Using the data of observed trade volume between pairs of countries
can lead to sample selection bias because the sample selection
process omit the pairs of countries that do not trade with each
other. But the fact that two countries do not trade may also bear
some economic meaning that can influence the causal effect of the
variables of interest on trade volume.
\end{itemize}
\end{itemize}

\begin{itemize}
\item Solutions to sample selection bias
\label{sec:org8c20699}
\begin{itemize}
\item Collect the sample in a way that avoids sample
\item Randomized controlled experiment.
\item Construct a model of the sample selection problem and estimate that
model.
\end{itemize}
\end{itemize}

\subsection{Simultaneous causality}
\label{sec:org31ecc57}
Up to now, all we examined is how \(X\) can cause \(Y\). What if \(Y\) causes
\(X\)? If \(Y\) does cause \(X\) in some way, there is \textbf{simultaneous
causality} problem, which lead to biased and inconsistent OLS
estimator. 

There are many examples of simultaneous causality in Economics. In the
paper of Acemuglou et al.(2000), \emph{The Colonial Origins of Comparative
Development: An Empirical Investigation}, the authors estimate the
effect of institutions on economic performance. However, the
simultaneous causality (or mutual causality) comes from the fact that
not only do good institutions promote economic performance, but also
countries with high GDP per capita can afford good institutions and
secure property rights, which in turn yield better economic
performance. 

Simultaneous causality leads to biased estimates of the effect of \(X\)
on \(Y\), referred to as \textbf{simultaneous causality bias}. We can express
the simultaneous causality using a simultaneous equations.
\begin{gather}
Y_i = \beta_0 + \beta_1 X_i + u_i \label{eq:sim-cau-1} \\
X_i = \gamma_0 + \gamma_1 Y_i + v_i \label{eq:sim-cau-2}
\end{gather}

Intuitively, simultaneous causality comes from the following facts. 
\begin{itemize}
\item Large \(u_i\) means large \(Y_i\), which implies large \(X_i\) (if
\(\gamma_1\) > 0).
\item This implies that \(u_i\) and \(X_i\) are correlated, i.e., \(\cov(X_i,
  u_i) \neq 0\).
\item Thus, the OLS estimator of \(\beta_1\) from merely estimating Equation
(\ref{eq:sim-cau-1}) is biased and inconsistent.
\end{itemize}

Formally, we can prove that \(\cov(X_i, u_i) \neq 0\), resulting in the
bias in the OLS estimator of \(\beta_1\). 
\begin{proof}
\begin{align*}
\cov(X_i, u_i) &= \cov(\gamma_0 + \gamma_1 Y_i + v_i, u_i) \\
&= \gamma_1\cov(Y_i, u_i) + \cov(v_i, u_i) (\text{ Assuming } \cov(v_i, u_i)=0) \\
&= \gamma_1\cov(\beta_0 + \beta_1 X_i + u_i, u_i) \\
&= \gamma_1\cov(X_i, u_i) + \gamma_1\sigma^2_u
\end{align*}
Solving for $\cov(X_i, u_i)$ yields the result 
$\cov(X_i, u_i) = \gamma_1 \sigma^2_u /(1-\gamma_1\beta_1)$, which is not
equal to zero unless $\gamma_1 = 0$, i.e., the simultaneous causality does
exist. 
\end{proof}

\subsubsection*{Solutions to simultaneous causality bias}
\label{sec:org0a95d1d}
\begin{enumerate}
\item Run a randomized controlled experiment.  Because \(X_i\) is chosen at
random by the experimenter, there is no feedback from the outcome
variable to \(Y_i\) (assuming perfect compliance).
\item Develop and estimate a complete model of both directions of
causality.  This is the idea behind many large macro models
(e.g. Federal Reserve Bank-US).  This is extremely difficult in
practice.
\item Use instrumental variables regression to estimate the causal effect
of interest (effect of X on Y, ignoring effect of Y on X)
\end{enumerate}

\subsection{Sources of inconsistency of OLS standard errors}
\label{sec:orgb87a080}
Inconsistent standard errors pose a different threat to internal
validity. Even if the OLS estimator is consistent and the sample is
large, inconsistent standard errors will produce hypothesis tests with
size that differs from the desired significance level and "95\%"
confidence intervals that fail to include the true value in 95\% of
repeated samples. 

There are two main reasons for inconsistent standard errors:
improperly handled heteroskedasticity and correlation of the error
term across observations.

\subsubsection*{Heteroskedasticity}
\label{sec:orgebe4d3b}
If the errors are heteroskedastic and you mistakenly use the
homoskedasticity-only standard errors that are reported by some
software by default, then the t-test and the F-test based on the wrong
standard errors do not have the desired size. 

The solution to this problem is to use heteroskedasticity-robust
standard errors of the OLS estimators and to construct t- and
F-statistics using a heteroskedasticity-robust variance estimator,
which is provided as an option in modern software packages. 

\begin{itemize}
\item \textbf{The Breusch-Pagan test for heteroskedasticity}
\label{sec:orgfdc45d0}

We can test whether heteroskedasticity exists in a regression model
using the Breusch-Pagan test. The test consist of the following steps: 
\begin{enumerate}
\item Estimate a regression model, \(Y = \beta_0 + \beta_1 X_1 + \cdots +
   \beta_k X_k + u\), and obtain the squared OLS residuals,
\(\hat{u}^2\).
\item Run a regression of \(\hat{u}^2 = \delta_0 + \delta_1 X_1 + \cdots +
   \delta_k X_k + v\), and obtain the \(R^2\) of this regression, denoted
as \(R^2_{\hat{u}^2}\).
\item Test the null hypothesis, \(H_0: E(u^2 | X_1, \ldots, X_k) =
   \sigma^2\), i.e., homoskedasticity, against the alternative
hypothesis for heteroskedasticity. The test statistics can be the
overall F statistics for the regression in the second step, which
is
\[ F = \frac{R^2_{\hat{u}^2}/k}{(1 - R^2_{\hat{u}^2})/(n-k-1)} \sim
   F(k, n-k-1)\]
Or we can compute an LM test statistics, which is
\[ LM = n R^2_{\hat{u}^2} \sim \chi^2(k) \]
where \(n\) is the number of observations.
\item Based on the F-statistic or the LM statistic, compute the
p-value. If the p-value is smaller than the significance level, we
can reject the null hypothesis of homoskedasticity.
\end{enumerate}
\end{itemize}

\subsubsection*{Correlation of the error term across observations}
\label{sec:orgd1075ec}
In the lease squares assumptions, we assume that \((X_i, Y_i)\) for
\(i=1, \ldots, n\) are i.i.d., which implies that \(u_i\) are uncorrelated
across observations. However, in some setting, the population
regression error can be correlated across observations. There are
mainly two types of correlation in consideration: serial correlation
and spatial correlation. 

\begin{itemize}
\item Serial correlation arises from the repeated observations over the
same entity over time. It is a prevalent problem in time series
data.
\item Spatial correlation arises from the influence of contiguous
(neighboring) observations over geographic units.
\item The OLS estimator with serial correlation or spatial correlation is
still unbiased and consistent, but inference based on no correlation
assumption is not valid.
\item Solution: 
\begin{itemize}
\item use the \textbf{heteroskedasticity-and-auto-correlation-consistent
standard errors (HAC)}. We will learn how to handle serial
correlation in time series data in the next two semesters.
\item Model the spatial correlation specifically. Spatial econometrics
is a branch of econometrics that deals with spatial correlation.
\end{itemize}
\end{itemize}
\end{document}