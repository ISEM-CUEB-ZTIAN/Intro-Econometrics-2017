% Created 2017-02-21 Tue 22:16
% Intended LaTeX compiler: pdflatex
\documentclass[presentation]{beamer}
\usepackage[utf8]{inputenc}
\usepackage[T1]{fontenc}
\usepackage{graphicx}
\usepackage{grffile}
\usepackage{longtable}
\usepackage{wrapfig}
\usepackage{rotating}
\usepackage[normalem]{ulem}
\usepackage{amsmath}
\usepackage{textcomp}
\usepackage{amssymb}
\usepackage{capt-of}
\usepackage{hyperref}
\usetheme{CambridgeUS}
\usecolortheme{beaver}
\setcounter{secnumdepth}{1}
\author{Zheng Tian}
\date{}
\title{Lecture 2: Review of Probability}
\hypersetup{
 pdfauthor={Zheng Tian},
 pdftitle={Lecture 2: Review of Probability},
 pdfkeywords={},
 pdfsubject={},
 pdfcreator={Emacs 25.1.1 (Org mode 9.0.3)}, 
 pdflang={English}}
\begin{document}

\maketitle
\begin{frame}{Outline}
\setcounter{tocdepth}{1}
\tableofcontents
\end{frame}


\section{Random Variables and Probability Distributions}
\label{sec:org8ec0b88}

\subsection*{Defining probabilities and random variables}
\label{sec:org878cd5d}

\begin{frame}[label={sec:org8ee8fe1}]{Experiments and outcomes}
\begin{itemize}
\item An \alert{experiment} is the processes that generate random results
\item The \alert{outcomes} of an experiment are its
mutually exclusive potential results.
\item Example: tossing a coin. The outcome is either getting a head(H) or a tail(T)
but not both.
\end{itemize}
\end{frame}

\begin{frame}[label={sec:orgd252c12}]{Sample space and events}
\begin{itemize}
\item A \alert{sample space} consists of all the outcomes from an experiment,
denoted with the set \(S\).
\begin{itemize}
\item \(S = \{H, T\}\) in the tossing-coin experiment.
\end{itemize}

\item An \alert{event} is a subset of the sample 
space.

\item Getting a head is an event, which is \(\{H\} \subset \{H, T\}\).
\end{itemize}
\end{frame}

\subsection*{Probability}
\label{sec:org92ad719}

\begin{frame}[label={sec:org47339db}]{An intuitive definition of probability}
\begin{itemize}
\item The \alert{probability} of an event is the proportion of the time that the
event will occur in the long run.

\item For example, we toss a coin for \(n\)
times and get \(m\) heads. When \(n\) is very large, we can say that the
probability of getting a head in a toss is \(m/n\).
\end{itemize}
\end{frame}

\begin{frame}[label={sec:org955455a}]{An axiomatic definition of probability}
\begin{itemize}
\item A probability of an event \(A\) in the sample space \(S\), denoted as
\(\mathrm{Pr}(A)\), is a function that assign \(A\) a real number in \([0,
  1]\), satisfying the following three conditions:
\begin{enumerate}
\item \(0 \leq \mathrm{Pr}(A) \leq 1\).
\item \(\mathrm{Pr}(S) = 1\).
\item For any disjoint sets, \(A\) and \(B\), that is \(A\) and \(B\) have no
element in common, \(\mathrm{Pr}(A \cup B) = \mathrm{Pr}(A) +
    \mathrm{Pr}(B)\).
\end{enumerate}
\end{itemize}
\end{frame}

\subsection*{Random variables}
\label{sec:org3872f74}

\begin{frame}[label={sec:orgd163cd6}]{The definition of random variables}
\begin{itemize}
\item A \alert{random variable} is a numerical summary associated with the
outcomes of an experiment.

\item You can also think of a random variable as a function
mapping from an event \(\omega\) in the sample space \(\Omega\) to the
real line.
\end{itemize}
\end{frame}

\begin{frame}[label={sec:org8061353}]{An illustration of random variables}
\begin{figure}[htbp]
\centering
\includegraphics[width=0.8\textwidth]{figure/random_variable_demo1.png}
\caption{\label{fig:org506f65b}
An illustration of random variable}
\end{figure}
\end{frame}

\begin{frame}[label={sec:org342d5cb}]{Discrete and continuous random variables}
Random variables can take different types of values

\begin{itemize}
\item A \alert{discrete} random
variables takes on a discrete set of values, like \(0, 1, 2, \ldots, n\)
\item A \alert{continuous} random variable takes on a continuum of possble
values, like any value in the interval \((a, b)\).
\end{itemize}
\end{frame}

\subsection*{Probability distributions}
\label{sec:org753bd6c}

\begin{frame}[label={sec:org9678599}]{The probability distribution for a discrete random variable}
\begin{itemize}
\item The probability distribution of a discrete random variable is the list
of all possible values of the variable and the probability that each
value will occur. These probabilities sum to 1.

\item The probability mass function. Let \(X\) be a discrete random
variable. The probability distribution of \(X\) (or the probability
mass function), \(p(x)\), is
\begin{equation*}
p(x) = \mathrm{Pr}(X = x)
\end{equation*}

\item The axioms of probability require that 
\begin{enumerate}
\item \(0 \leq p(x) \leq  1\)
\item 2) \(\sum_{i=1}^n p(x_i) =  1\).
\end{enumerate}
\end{itemize}
\end{frame}

\begin{frame}[label={sec:org37bba7d}]{An example of the probability distribution of a discrete random variable}
\begin{table}[htbp]
\caption{\label{tab:org388079b}
An illustration of the probability distribution of a discrete random variable}
\centering
\begin{tabular}{lrrrr}
\toprule
\(X\) & 1 & 2 & 3 & Sum\\
\midrule
\(\mathrm{P}(x)\) & 0.25 & 0.75 & 0.25 & 1\\
\bottomrule
\end{tabular}
\end{table}
\end{frame}

\subsection*{The cumulative probability distribution}
\label{sec:org5d18725}

\begin{frame}[label={sec:org1d8bd9b}]{Definition of the c.d.f.}
\begin{itemize}
\item The \alert{cumulative probability distribution} (or the cumulative
distribution function, c.d.f.): 

Let \(F(x)\) be the c.d.f of \(X\). Then \(F(x) = \mathrm{Pr}(X \leq x)\).
\end{itemize}

\begin{table}[htbp]
\caption{\label{tab:orgbe2a6a9}
An illustration of the c.d.f. of a discrete random variable}
\centering
\begin{tabular}{lrrrl}
\toprule
\(X\) & 1 & 2 & 3 & Sum\\
\midrule
\(\mathrm{P}(x)\) & 0.25 & 0.50 & 0.25 & 1\\
C.d.f. & 0.25 & 0.75 & 1 & --\\
\bottomrule
\end{tabular}
\end{table}
\end{frame}

\begin{frame}[label={sec:orge67a82a}]{An illustration of the c.d.f. of a discrete random variable}
\begin{figure}[htbp]
\centering
\includegraphics[width=0.53\textwidth,height=0.45\textheight]{figure/cdf_discrete_example.png}
\caption{\label{fig:orgbc500e5}
The c.d.f. of a discrete random variable}
\end{figure}
\end{frame}

\begin{frame}[label={sec:orgc432871}]{Bernouli distribution}
The Bernoulli distribution
\begin{equation*}
  G =
    \begin{cases}
      1 & \text{with probability } p \\
      0 & \text{with probability } 1-p
    \end{cases}
  \end{equation*}
\end{frame}

\subsection*{The probability distribution of a continuous random variable}
\label{sec:org2e4aee7}

\begin{frame}[label={sec:org3eff1e4}]{Definition of the c.d.f. and the p.d.f.}
\begin{itemize}
\item The cumulative distribution function of a continous random variable
is defined as it is for a discrete random variable. 
\[ F(x) = \mathrm{Pr}(X \leq x) \]

\item The \alert{probability density function (p.d.f.)} of \(X\) is the function
that satisfies
\[ F(x) = \int_{-\infty}^{x} f(t) \mathrm{d}t \text{ for all } x \]
\end{itemize}
\end{frame}

\begin{frame}[label={sec:org7d1eb83}]{Properties of the c.d.f.}
\begin{itemize}
\item For both discrete and continuous random variable, \(F(X)\) must satisfy
the following properties:
\begin{enumerate}
\item \(F(+\infty) = 1 \text{ and } F(-\infty) = 0\) (\(F(x)\) is bounded between 0 and 1)
\item \(x > y \Rightarrow F(x) \geq F(y)\) (\(F(x)\) is nondecreasing)
\end{enumerate}

\item By the definition of the c.d.f., we can conveniently calculate
probabilities, such as,
\begin{itemize}
\item \(\mathrm{P}(x > a) = 1 - \mathrm{P}(x \leq a) = 1 - F(a)\)
\item \(\mathrm{P}(a < x \leq b) = F(b) - F(a)\).
\end{itemize}
\end{itemize}
\end{frame}

\begin{frame}[label={sec:orgb2c33a4}]{The c.d.f. and p.d.f. of a normal distribution}
\begin{figure}[htbp]
\centering
\includegraphics[width=0.6\textwidth,height=0.5\textheight]{figure/norm1.png}
\caption{\label{fig:org4814d46}
The p.d.f. and c.d.f. of a continuous random variable (the normal distribution)}
\end{figure}
\end{frame}


\section{Expectation, Variance, and Other Moments}
\label{sec:org3da791b}

\subsection*{The expected value of a random variable}
\label{sec:org368edef}

\begin{frame}[label={sec:org461442c}]{The expected value}
\begin{itemize}
\item The \alert{expected value} of a random variable, X, denoted as \(\mathrm{E}(X)\), is
the long-run average of the random variable over many repeated
trials or occurrences, which is also called the \alert{expectation} or the
\alert{mean}.

\item The expected value measures the centrality of a random variable.
\end{itemize}
\end{frame}

\begin{frame}[label={sec:orgb1b651d}]{Mathematical definition}
\begin{itemize}
\item For a discrete random variable
\[ \mathrm{E}(X) = \sum_{i=1}^n x_i \mathrm{Pr}(X = x_i) \]

\item e.g. The expectation of a Bernoulli random variable, \(G\),
\[ \mathrm{E}(G) = 1 \cdot p + 0 \cdot (1-p) = p \]

\item For a continuous random variable
\[ \mathrm{E}(X) = \int_{-\infty}^{\infty} x f(x) \mathrm{d}x\]
\end{itemize}
\end{frame}

\subsection*{The variance and standard deviation}
\label{sec:org847377e}

\begin{frame}[label={sec:orgcd0b0fe}]{Definition of variance and standard deviation}
\begin{itemize}
\item The \alert{variance} of a random variable \(X\) measures its average
deviation from its own expected value.

\item Let \(\mathrm{E}(X) = \mu_X\). Then the variance of \(X\),

\begin{align*}
\mathrm{Var}(X) & =  \sigma^2_X =  \mathrm{E}(X-\mu_X)^{2} \\
& = 
\begin{cases}
\sum_{i=1}^n (x - \mu_X)^{2}\mathrm{Pr}(X = x_i) & \text{if } X \text{ is discrete} \\
\int_{-\infty}^{\infty} (x - \mu_X)^{2}f(x)\mathrm{d} x  & \text{if } X \text{ is continuous}
\end{cases}
\end{align*}

\item The \alert{standard deviation} of \(X\): \(\sigma_{X} = \sqrt{\mathrm{Var}(X)}\)
\end{itemize}
\end{frame}

\begin{frame}[label={sec:orgbd52c9e}]{Computing variance}
\begin{itemize}
\item A convenient formula for calculating the variance is
\[ \mathrm{Var}(X) = \mathrm{E}(X - \mu_X)^{2} = \mathrm{E}(X^{2}) - \mu_X^{2} \]

\item The variance of a Bernoulli random variable, \(G\)
\[ \mathrm{Var}(G) = (1-p)^{2}p + (0-p)^{2}(1-p) = p(1-p) \]

\item The expectation and variance of a linear function of \(X\). Let \(Y = a +
  bX\), then
\begin{itemize}
\item \(\mathrm{E}(Y) = a + \mathrm{E}(X)\)
\item \(\mathrm{Var}(Y) = \mathrm{Var}(a + b X) = b^{2} \mathrm{Var}(X)\).
\end{itemize}
\end{itemize}
\end{frame}

\subsection*{Moments of a random variable, skewness and kurtosis}
\label{sec:orgdb62186}

\begin{frame}[label={sec:org9f88054}]{Definition of the moments of a distribution}
\begin{description}
\item[{k\(^{\text{th}}\) moment}] The k\(^{\text{th}}\) \alert{moment} of the distribution of \(X\) is
\(\mathrm{E}(X^{k})\). So, the expectation is the "first"
moment of \(X\).

\item[{k\(^{\text{th}}\) central moment}] The k\(^{\text{th}}\) central moment of the distribution
of \(X\) with its mean \(\mu_X\) is \(\mathrm{E}(X - \mu_X)^{k}\). So, the
variance is the second central moment of \(X\).
\end{description}

\begin{block}{A caveat}
It is important to remember that not all the moments of a distribution
exist. 
\end{block}
\end{frame}

\begin{frame}[label={sec:org8264f6f}]{Skewness}
\begin{itemize}
\item The skewness of a distribution provides a mathematical way to describe
how much a distribution deviates from symmetry.

\[ \text{Skewness} =  \mathrm{E}(X - \mu_X)^{3}/\sigma_{X}^{3} \]

\item A symmetric distribution has a skewness of zero.
\item The skewness can be either positive or negative.
\item That \(\mathrm{E}(X - \mu_X)^3\) is divided by \(\sigma^3_X\) is to make
the skewness measure unit free.
\end{itemize}
\end{frame}

\begin{frame}[label={sec:org5dd5889}]{Kurtosis}
\begin{itemize}
\item The kurtosis of the distribution of a random variable \(X\) measures how
much of the variance of \(X\) arises from extreme values, which makes
the distribution have "heavy" tails.

\[ \text{Kurtosis} = \mathrm{E}(X - \mu_X)^{4}/\sigma_{X}^{4} \]

\item The kurtosis must be positive.
\item The kurtosis of the normal distribution is 3. So a distribution that
has its kurtosis exceeding 3 is called heavy-tailed.
\item The kurtosis is also unit free.
\end{itemize}
\end{frame}

\begin{frame}[label={sec:org4c75dbb}]{An illustration of skewness and kurtosis}
\begin{center}
\includegraphics[width=0.5\textwidth,height=0.5\textheight]{figure/fig-2-3.png}
\end{center}

\begin{itemize}
\item All four distributions have a mean of zero and
a variance of one, while (a) and (b) are symmetric and (b)-(d) are
heavy-tailed.
\end{itemize}
\end{frame}


\section{Two Random Variables}
\label{sec:orgabc3c9c}

\begin{frame}[label={sec:org7baf345}]{The joint and marginal distributions}
\begin{block}{The joint probability function of two discrete random variables}
\begin{itemize}
\item The joint distribution of two random variables \(X\) and \(Y\) is
\[ p(x, y) = \mathrm{Pr}(X = x, Y = y)\]

\item \(p(x, y)\) must satisfy
\begin{enumerate}
\item \(p(x, y) \geq 0\)
\item \(\sum_{i=1}^n\sum_{j=1}^m p(x_i, y_j) = 1\) for all possible
combinations of values of \(X\) and \(Y\).
\end{enumerate}
\end{itemize}
\end{block}

\begin{block}{The joint probability function of two continuous random variables}
\begin{itemize}
\item For two continuous random variables, \(X\) and \(Y\), the counterpart of \(p(x, y)\) is
the joint probability density function, \(f(x, y)\), such that
\begin{enumerate}
\item \(f(x, y) \geq 0\)
\item \(\int_{-\infty}^{{\infty}} \int_{-\infty}^{\infty} f(x, y)\, dx\, dy= 1\)
\end{enumerate}
\end{itemize}
\end{block}
\end{frame}

\begin{frame}[label={sec:org2d0eb51}]{The marginal probability distribution}
\begin{itemize}
\item The marginal probability distribution of a random variable \(X\) is
simply the probability distribution of its own.

\item For a discrete random variable, we can compute the marginal
distribution of \(X\) as
\[ \mathrm{Pr}(X=x) = \sum_{i=1}^n \mathrm{Pr}(X, Y=y_i) = \sum_{i=1}^n p(x, y_i)  \]

\item For a continuous random variable, the marginal distribution is
\[f_X(x) = \int_{-\infty}^{\infty} f(x, y)\, dy \]
\end{itemize}
\end{frame}

\begin{frame}[label={sec:org813b02d}]{An example of joint and marginal distributions}
\begin{table}[htbp]
\caption{\label{tab:org5353127}
Joint and marginal distributions of raining and commuting time}
\centering
\begin{tabular}{lrrr}
 & Rain (\(X=0\)) & No rain (\(X=1\)) & Total\\
\hline
Long commute (\(Y=0\)) & 0.15 & 0.07 & 0.22\\
Short commute (\(Y=1\)) & 0.15 & 0.63 & 0.78\\
\hline
Total & 0.30 & 0.70 & 1\\
\end{tabular}
\end{table}
\end{frame}

\begin{frame}[label={sec:org0b3a35a}]{Conditional probability}
\begin{itemize}
\item For any two events \(A\) and \(B\), the conditional probability of \(A\) given
\(B\) is defined as
\begin{equation*}
\mathrm{Pr}(A|B) = \frac{\mathrm{Pr}(A \cap B)}{\mathrm{Pr}(B)}
\end{equation*}
\end{itemize}

\begin{center}
\includegraphics[width=0.4\textwidth,height=0.4\textheight]{figure/conditional_probability.png}
\end{center}
\end{frame}

\begin{frame}[label={sec:orga98c3ed}]{The conditional probability distribution}
\begin{itemize}
\item The conditional distribution of a random variable \(Y\) given another
random variable \(X\) is \(\mathrm{Pr}(Y | X=x)\).

\item The formula to compute it is
\[ \mathrm{Pr}(Y | X=x) = \frac{\mathrm{Pr}(X=x, Y)}{\mathrm{Pr}(X=x)} \]

\item For continuous random variables \(X\) and \(Y\), we define the conditional
density function as
\[ f(y|x) = \frac{f(x, y)}{f_X(x)} \]
\end{itemize}
\end{frame}

\begin{frame}[label={sec:org73c9778}]{The conditional expectation}
\begin{itemize}
\item The \alert{conditional expectation} of \(Y\) given \(X\) is the expected value
of the conditional distribution of \(Y\) given \(X\).

\item For discrete random variables, the conditional mean of \(Y\) given \(X=x\) is
\begin{equation*}
\mathrm{E}(Y \mid X=x) = \sum_{i=1}^n y_i \mathrm{Pr}(Y \mid X=x)
\end{equation*}

\item For continuous random variables, it is computed as
\begin{equation*}
\int_{-\infty}^{\infty} y f(y \mid x)\, dy
\end{equation*}

\item The expected mean of commuting time given it is raining is \(0 \times
  0.1 + 1 \times 0.9 = 0.9\).
\end{itemize}
\end{frame}

\begin{frame}[label={sec:orgb4c8e1b}]{The law of iterated expectation}
\begin{itemize}
\item \alert{The law of iterated expectation}:

\[ \mathrm{E}(Y) = E \left[ \mathrm{E}(Y|X) \right] \]

\item It says that the mean of \(Y\) is the weighted average of the
conditional expectation of \(Y\) given \(X\), weighted by the
probability distribution of \(X\). That is,
\[ \mathrm{E}(Y) = \sum_{i=1}^n \mathrm{E}(Y \mid X=x_i) \mathrm{Pr}(X=x_i) \]

\item If \(\mathrm{E}(X|Y) = 0\), then \(\mathrm{E}(X)=E\left[\mathrm{E}(X|Y)\right]=0\).
\end{itemize}
\end{frame}

\begin{frame}[label={sec:org4874f94}]{Conditional variance}
\begin{itemize}
\item With the conditional mean of \(Y\) given \(X\), we can compute the
conditional variance as
\[ \mathrm{Var}(Y \mid X=x) = \sum_{i=1}^n \left[ y_i - \mathrm{E}(Y \mid X=x)
  \right]^2 \mathrm{Pr}(Y=y_i \mid X=x) \]

\item From the law of iterated expectation, we can get the following
\[ \mathrm{Var}(Y) = \mathrm{E}(\mathrm{Var}(Y \mid X)) + \mathrm{Var}(\mathrm{E}(Y \mid
  X)) \]
\end{itemize}
\end{frame}

\begin{frame}[label={sec:org250ba8b}]{Independent random variables}
\begin{itemize}
\item Two random variables \(X\) and \(Y\) are \alert{independently distributed}, or
\alert{independent}, if knowing the value of one of the variable provides no
information about the other.
\item Mathematically, it means that 
\[ \mathrm{Pr}(Y=y \mid X=x) = \mathrm{Pr}(Y=y)  \]

\item If \(X\) and \(Y\) are independent
\[ \mathrm{Pr}(Y=y, X=x) = \mathrm{Pr}(X=x) \mathrm{Pr}(Y=y) \]
\end{itemize}
\end{frame}

\begin{frame}[label={sec:org5a655c6}]{Independence between two continuous random variable}
\begin{itemize}
\item For two continuous random variables, \(X\) and \(Y\), they are
\alert{independent} if
\[ f(x|y) = f_{X}(x) \text{ or } f(y|x) = f_{Y}(y) \]

\item It follows that if \(X\) and \(Y\) are independent
\[ f(x, y) = f(x|y)f_{Y}(y) = f_{X}(x)f_{Y}(y) \]
\end{itemize}
\end{frame}

\subsection*{Covariance and Correlation}
\label{sec:orgd59126b}

\begin{frame}[label={sec:orgcc34837}]{Covariance}
\begin{itemize}
\item The covariance of two discrete random variables \(X\) and \(Y\) is
\begin{align*}
\mathrm{Cov}(X, Y) & = \sigma_{XY} = \mathrm{E}(X-\mu_{X})(Y-\mu_{Y}) \\
                   & = \sum_{i=1}^n \sum_{j=1}^m (x_i - \mu_X)(y_j - \mu_Y) \mathrm{Pr}(X=x_i, Y=y_j)
\end{align*}

\item For continous random variables, the covariance of \(X\) and \(Y\) is
\[ \mathrm{Cov}(X, Y) = \int_{-\infty}^{\infty}
  \int_{-\infty}^{\infty} (x-\mu_X)(y-\mu_y)f(x, y) dx dy \]

\item The covariance can also be computed as
\[ \mathrm{Cov}(X, Y) = \mathrm{E}(XY) - \mathrm{E}(X)\mathrm{E}(Y) \]
\end{itemize}
\end{frame}

\begin{frame}[label={sec:org6697b72}]{Correlation coefficient}
\begin{itemize}
\item The \alert{correlation coefficient} of \(X\) and \(Y\) is

\[ \mathrm{corr}(X, Y) = \rho_{XY} = \frac{\mathrm{Cov}(X, Y)}{\left[\mathrm{Var}(X)\mathrm{Var}(Y)\right]^{1/2}} =
  \frac{\sigma_{XY}}{\sigma_{X}\sigma_{Y}} \]

\item \(-1 \leq \mathrm{corr}(X, Y) \leq 1\).

\item \(\mathrm{corr}(X, Y)=0\) (or \(\mathrm{Cov}(X,Y)=0\)) means that \(X\)
and \(Y\) are uncorrelated.

\item Since \(\mathrm{Cov}(X, Y) = \mathrm{E}(XY) -
  \mathrm{E}(X)\mathrm{E}(Y)\), when \(X\) and \(Y\) are uncorrelated, then \(\mathrm{E}(XY) =
  \mathrm{E}(X) \mathrm{E}(Y)\).
\end{itemize}
\end{frame}

\begin{frame}[label={sec:org10fda8a}]{Independence and uncorrelation}
\begin{itemize}
\item If \(X\) and \(Y\) are independent, then
\begin{align*}
\mathrm{Cov}(X, Y) & = \sum_{i=1}^n \sum_{j=1}^m (x_i - \mu_X)(y_j - \mu_Y) \mathrm{Pr}(X=x_i) \mathrm{Pr}(Y=y_j) \\
                   & = \sum_{i=1}^n (x_i - \mu_X) \mathrm{Pr}(X=x_i) \sum_{j=1}^m (y_j - \mu_y) \mathrm{Pr}(Y=y_j) \\
                   & = 0 \times 0 = 0
\end{align*}

\item That is, if \(X\) and \(Y\) are independent, they must be
uncorrelated.

\item However, the converse is not true. If \(X\) and \(Y\) are
uncorrelated, there is a possibility that they are actually
dependent.
\end{itemize}
\end{frame}

\begin{frame}[label={sec:org74da8c6}]{Conditional mean and correlation}
\begin{itemize}
\item If \(X\) and \(Y\) are independent, then we must have 
\(\mathrm{E}(Y \mid X) = \mathrm{E}(Y) = \mu_Y\)

\item Then, we can prove that
\(\mathrm{Cov}(X, Y) = 0\) and \(\mathrm{corr}(X, Y)=0\).

\begin{align*}
\mathrm{E}(XY) & = \mathrm{E}(\mathrm{E}(XY \mid X)) = \mathrm{E}(X \mathrm{E}(Y \mid X)) \\
               & = \mathrm{E}(X) \mathrm{E}(Y \mid X) = \mathrm{E}(X) \mathrm{E}(Y)
\end{align*}

It follows that \(\mathrm{Cov}(X,Y) = \mathrm{E}(XY) - \mathrm{E}(X)
   \mathrm{E}(Y) = 0\) and \(\mathrm{corr}(X, Y)=0\).
\end{itemize}
\end{frame}

\begin{frame}[label={sec:org19713f5}]{Some useful operations}
The following properties
of \(\mathrm{E}(\cdot)\), \(\mathrm{Var}(\cdot)\) and
\(\mathrm{Cov}(\cdot)\) are useful in calculation,

\begin{align*}
\mathrm{E}(a + bX + cY)      & = a + b \mu_{X} + c \mu_{Y} \\
\mathrm{Var}(aX + bY)        & = a^{2} \sigma^{2}_{X} + b^{2} \sigma^{2}_{Y} + 2ab\sigma_{XY} \\
\mathrm{Cov}(a + bX + cV, Y) & = b\sigma_{XY} + c\sigma_{VY} \\
\end{align*}
\end{frame}
\end{document}