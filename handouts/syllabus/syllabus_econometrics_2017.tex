% Created 2017-01-24 Tue 20:29
% Intended LaTeX compiler: pdflatex
\documentclass[a4paper,11pt]{article}
\usepackage[utf8]{inputenc}
\usepackage[T1]{fontenc}
\usepackage{graphicx}
\usepackage{grffile}
\usepackage{longtable}
\usepackage{wrapfig}
\usepackage{rotating}
\usepackage[normalem]{ulem}
\usepackage{amsmath}
\usepackage{textcomp}
\usepackage{amssymb}
\usepackage{capt-of}
\usepackage{hyperref}
\usepackage[margin=1in]{geometry}
\usepackage{setspace}
\singlespace
\usepackage{CJK}
\setcounter{secnumdepth}{1}
\author{Zheng Tian}
\date{Spring semester, 2017}
\title{Syllabus for Introduction to Econometrics}
\hypersetup{
 pdfauthor={Zheng Tian},
 pdftitle={Syllabus for Introduction to Econometrics},
 pdfkeywords={},
 pdfsubject={},
 pdfcreator={Emacs 25.1.1 (Org mode 9.0.3)},
 pdflang={English}}
\begin{document}

\maketitle

\section{Basic information}
\label{sec:orgf8ce1fa}
\subsection*{Time and location}
\label{sec:org0f9e4bf}

\begin{center}
\begin{tabular}{llll}
Odd weeks & Monday & 08:00 am -- 09:50 am & Buoxue Building (\begin{CJK}{UTF8}{gbsn}博学楼\end{CJK}) 706\\
 & Wednesday & 10:00 am -- 11:50 am & Buoxue Building (\begin{CJK}{UTF8}{gbsn}博学楼\end{CJK}) 306\\
Even weeks & Monday & 08:00 am -- 11:50 am & Buoxue Building (\begin{CJK}{UTF8}{gbsn}博学楼\end{CJK}) 706\\
\end{tabular}
\end{center}


\subsection*{Instructor information}
\label{sec:orgc211ace}

\begin{center}
\begin{tabular}{ll}
Name: & Zheng Tian (\begin{CJK}{UTF8}{gbsn}田峥\end{CJK})\\
Email: & ztian\_cueb@163.com\\
Office: & Angong Building (\begin{CJK}{UTF8}{gbsn}安工楼\end{CJK}) 215\\
Tel: & 83951054\\
\end{tabular}
\end{center}


\subsection*{Office hours}
\label{sec:org2830bcd}
Office hours are tentatively scheduled as follows,

\begin{center}
\begin{tabular}{ll}
Tuesday & 9:30 am -- 11:30 am\\
Friday & 9:30 am -- 11:30 am\\
\end{tabular}
\end{center}

You are welcomed to stop by our offices at any other time. But making
an appointment by email or phone in advance is highly recommended.


\section{Course description}
\label{sec:org5d5b15d}
\subsection*{Objectives}
\label{sec:org8856ba7}

This course is an introductory Econometrics course. Econometrics is a
subject consisting of "economic tricks" for quantitative analysis,
which is an indispensable component of economists' research
toolbox. The roles of Econometrics include, but not limited to,
quantitatively examining the relationship between various economic
variables, giving empirical support to economic theories, making
forecasts about economic performance in the future, and evaluating
policy impacts, etc. At an introductory level, the goals of this
course concern (1) letting students know basic econometric methods and
theories, and (2) enabling students to use software, primarily R, to
estimate a simple econometric model regarding their own research
interests.

Serving for these ends, the contents of this course cover, but not
limited to, the single and multiple OLS regression estimation,
hypothesis testing, model specification assessment, detection and
solutions to problems in regression, for example, heteroskedasticity
and multicollinearity, and panel data model. If time permitted, we may
also cover such topics as instrumental variable methods, and the
Probit and Logit models for a limited dependent variable. Except for
the concept of serial correlation, this course will not cover time
series econometrics, which is the main topic of the Econometric course
in the next semester.


\subsection*{Prerequisite}
\label{sec:orgf4a6c41}

Principles of Microeconomics and Macroeconomics, Calculus,
Introductory Probability and Statistics, and Linear Algebra.


\section{Textbooks}
\label{sec:orgd2f3166}
\subsection*{Required}
\label{sec:org0c107ce}

Stock, J. and Wastson, M. (2010) \emph{Introduction to Econometrics}, 3rd
edition
\href{http://www.amazon.cn/gp/product/B00R7EEEUY/ref\%3Dox\_sc\_act\_title\_2?ie\%3DUTF8\&psc\%3D1\&smid\%3DA1AJ19PSB66TGU}{\url{http://www.amazon.cn/gp/product/B00R7EEEUY/ref=ox\_sc\_act\_title\_2?ie=UTF8\&psc=1\&smid=A1AJ19PSB66TGU}}

\subsection*{Recommended}
\label{sec:org8df8d5e}

\begin{itemize}
\item Jeffery Wooldridge (2012) \emph{Introductory Econometrics}, 5th edition, China edition
\url{http://www.amazon.cn/dp/B00ITGHEYW/ref=pd\_bxgy\_14\_img\_2?ie=UTF8\&refRID=1HC25MXFFYG5GNPRFDS1}
\item Kleiber and Zeileis (2008) \emph{Applied Econometrics with R}
\end{itemize}


\section{Course materials}
\label{sec:org6805e6d}
\subsection*{Lecture notes}
\label{sec:orge17fdee}

Lecture notes will be sent via email and uploaded
in Baidu Cloud. The authorized links to lecture notes and other
materials will be sent via email when they are uploaded. Hence, you
must provide me your valid email addresses for the purposes of
communication and distributing course materials.

\textbf{Please read carefully lecture notes, which are the basis for all assignments and tests in this course.}


\subsection*{Book companion materials}
\label{sec:org1c72773}

There is a companion website for this book,
\url{http://wps.aw.com/aw\_stock\_ie\_3/}, where you can download dataset for
exercises, pratical quizzes, and STATA tutorial. You can also download
datasets for empirical homework in Baidu Cloud.


\section{Course assignments}
\label{sec:orgaebc4f8}
\subsection*{{\bfseries\sffamily TODO} Homework}
\label{sec:orgefd882c}

\begin{itemize}
\item Homework will be assigned every other week. Homework will help
you understand fundamental concepts in econometric theories and
grasp basic estimation and testing methods through practice and
applications.
\item Questions of homework will be selected from the end-of-chapter
exercises in the textbook of Stock and Watson. I suggest that you
read through the chapter(s) covered in the homework before answering
questions.
\item You can finish your homework by either handwriting or typesetting
using word process software, e.g., Microsoft Word, LaTex, and the
like. Typesetting is highly recommended.
\item Each homework set will be assigned on Tuesdays of odd weeks (except
for Homework 1 assigned on Week 2), and
\textbf{due at 12:00 am on Mondays} of the immediately following
week. You have one week to complete each homework set, and I
strongly suggest you not wait until the last minute before the due
time to complete it.
\item You can turn in your homework in class on Mondays or email to me
by the due time. Do not forget put your name on your homework.
\item Homework will be graded as A, B, C, and D, based on the following
rule
\begin{itemize}
\item A: homework is submitted by the due time. Numeric and mathematical
answers are correct for all questions with only minor
mistakes. Explanations to your answers are convincing with correct
use of econometric terminology. English writing is clear and
grammatically right. (A = 100 percent points)
\item B: homework is submitted by the due time. Numeric and mathematical
answers are correct for most questions. Explanations are sound but
may not be totally right. English writing is a little obscure with
minor grammatical errors. (B = 85 percent points)
\item C: homework is submitted by the due time. Numeric and mathematical
answers are correct for almost half of questions. Explanations may
not be right but correctly using related econometric
terminology. English writing is just understandable with obvious
grammatical errors. (C = 70 percent points)
\item D: homework is submitted by the due time. Numeric and mathematical
answers are correct only for a few questions. Explanations are
wrong but with some merits. English writing is poor in both
understanding and grammar. (D = 60 percent points)
\end{itemize}
\item Homework must be submitted by the due time. A grace period of late
submission can be granted by request in advance. If granted, you
must turn in your homework two days after the due day. Late
submission of homework is subject to downgrading score to a lower
level. No submission at all will result in no score on the homework.
\end{itemize}

\begin{itemize}
\item {\bfseries\sffamily TODO} Add a policy for group work in homework
\label{sec:orgc4fe5d8}
\end{itemize}


\subsection*{Mid-term examination}
\label{sec:org5be3276}

\begin{itemize}
\item The mid-term exam will cover most materials taught from Week 5 to
Week 9.
\item It is tentatively scheduled on \textbf{May 9th, Monday}.
\item This will be a closed-book test. But you are allowed to bring a
one-sided "cheat sheet", on which you can write down some notes that
help you remember some important definitions and formulas. You are
allowed to write on \textbf{only one side} on the cheat sheet.
\item If you miss the mid-term exam, a make-up test can be arranged. You
must notify me in advance of your absence with a valid excuse.
\end{itemize}


\subsection*{Final examination}
\label{sec:org6426d94}

\begin{itemize}
\item The final exam will be comprehensive, covering all being taught
throughout the semester.
\item The time and location are to be arranged and announced by the
university.
\item This will also be a closed-book test. But you are still allowed to
bring a "cheat sheet" written on \textbf{both sides} this time.
\item The make-up test will follow the rule of the university.
\end{itemize}


\subsection*{{\bfseries\sffamily TODO} Add course project description}
\label{sec:org24255ff}


\section{Grade distribution}
\label{sec:org7d450eb}

\begin{center}
\begin{tabular}{lr}
Assignments & Scores\\
\hline
Homework & 20\\
Course project & 10\\
Midterm exam & 30\\
Final exam & 40\\
\hline
total & 100\\
\end{tabular}
\end{center}

A total of five bonus points will be offered contingent on your
performance. The announcement about the chances earning bonus points
will be made in the class.


\section{Course outline and schedule}
\label{sec:orgd7b892e}

Table \ref{tab:org978aebc} is the tentative outline and schedule for this course.\footnote{The instructor reserves the right to change this syllabus as
time and circumstances dictate. Necessary changes will be announced in
class in advance when possible.} The
schedule is subject to change according to the actual course
progress. Chapters referred in the table are in the required
textbook. Other related references would be cited in lecture notes.

{\small
\begin{longtable}{p{2.8cm}p{9cm}p{3cm}}
\caption{\label{tab:org978aebc}
Tentative Course Schedule}
\\
 &  & \\
Dates & Contents & Due dates\\
\hline
\endfirsthead
\multicolumn{3}{l}{Continued from previous page} \\

Dates & Contents & Due dates \\

\hline
\endhead
\hline\multicolumn{3}{r}{Continued on next page} \\
\endfoot
\endlastfoot
\hline
Week 1 &  & \\
\textit{[2017-02-20 Mon]} & Syllabus and Introduction (Chapter 1) & \\
\textit{[2017-02-22 Wed]} & Review of probability (Chapter 2) & \\
\hline
Week 2 &  & \\
\textit{[2017-02-27 Mon]} & Review of statistics (Chapter 3) & \\
\hline
Week 3 &  & \\
\textit{[2017-03-06 Mon]} & Continue review of statistics & Homework 1 due\\
\textit{[2017-03-08 Wed]} & Review of linear algebra (Appendix 18.1) & \\
\hline
Week 4 &  & \\
\textit{[2017-03-13 Mon]} & Introduction to R & \\
\hline
Week 5 &  & \\
\textit{[2017-03-20 Mon]} & Single regression: estimation (Chapters 4 and 17) & \\
\textit{[2017-03-22 Wed]} & Continue on single regression estimation & \\
\hline
Week 6 &  & \\
\textit{[2017-04-03 Mon]} & Single regression: hypothesis tests (chapters 5 and 17) & Homework 2 due\\
\hline
Week 7 &  & \\
\textit{[2017-04-10 Mon]} & Continue on single regression hypothesis tests & \\
\textit{[2017-04-12 Wed]} & Single regression with R and introduction to R Markdown & \\
\hline
Week 8 &  & \\
\textit{[2017-04-17 Mon]} & Multiple regression: estimation (chapters 6 and 18) & Homework 3 due\\
\hline
Week 9 &  & \\
\textit{[2017-04-24 Mon]} & Continue on multiple regression estimation & \\
\textit{[2017-04-26 Wed]} & Multiple regression: hypothesis tests (chapters 7 and 18) & \\
\hline
Week 10 &  & \\
\textit{[2017-05-01 Mon]} & Labor Day break & Homework 4 due\\
\hline
Week 11 &  & \\
\textit{[2017-05-08 Mon]} & Mid-term examination & \\
\textit{[2017-05-10 Wed]} & Continue on multiple regression hypothesis tests & \\
\hline
Week 12 &  & \\
\textit{[2017-05-15 Mon]} & Multiple regression with R & Homework 5 due\\
\hline
Week 13 &  & \\
\textit{[2017-05-22 Mon]} & Nonlinear regressions (chapter 8) & \\
\textit{[2017-05-24 Wed]} & Continue on nonlinear regressions and R & \\
\hline
Week 14 &  & \\
\textit{[2017-05-29 Mon]} & Assessing multiple regression (chapter 9) & Homework 6 due\\
\hline
Week 15 &  & \\
\textit{[2017-06-05 Mon]} & Continue on assessing multiple regression & \\
\textit{[2017-06-07 Wed]} & Regression with panel data (chapter 10) & \\
\hline
Week 16 &  & \\
\textit{[2017-06-12 Mon]} & Continue on panel data model and R & Homework 7 due\\
\hline
Week 17 &  & \\
TBA & Final examination & \\
\hline
\end{longtable}
}


\section{Policy on academic dishonesty}
\label{sec:orga93f11d}

Academic dishonesty is defined to include but is not limited to the
following: plagiarism; cheating and dishonest practices in connection
with examinations, papers and projects; forgery, misrepresentation and
fraud. Such behavior will not be tolerated and will be handled
according to university guidelines.
\end{document}